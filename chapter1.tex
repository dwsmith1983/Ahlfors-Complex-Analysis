\chapter{Complex Numbers}

\section{The Algebra of Complex Numbers}

\subsection{Arithmetic Operations}

\begin{enumerate}
\item
  Find the values of
  \[
  (1 + 2i)^3, \qquad \frac{5}{-3 + 4i}, \qquad
  \Bigl(\frac{2 + i}{3 - 2i}\Bigr)^2, \qquad (1 + i)^n + (1 - i)^n
  \]
  For the first problem, we have \((1 + 2i)^3 = (-3 + 4i)(1 + 2i) = -11 - 2i\).
  For the second problem, we should multiple by the conjugate
  \(\bar{z} = -3 - 4i\).
  \[
  \frac{5}{-3 + 4i}\frac{-3 - 4i}{-3 - 4i} = \frac{-15 - 20i}{25} =
  \frac{-3}{5} - \frac{4}{5}i
  \]
  For the third problem, we should first multiple by \(\bar{z} = 3 + 2i\).
  \[
  \frac{2 + i}{3 - 2i}\frac{3 + 2i}{3 + 2i} = \frac{8 + i}{13}
  \]
  Now we need to just square the result.
  \[
  \frac{1}{169}(8 + i)^2 = \frac{63 + 16i}{169}
  \]
  For the last problem, we will need to find the polar form of the complex
  numbers.
  Let \(z_1 = 1 + i\) and \(z_2 = 1 - i\).
  Then the modulus of \(z_1 = \sqrt{2} = z_2\).
  Let \(\phi_1\) and \(\phi_2\) be the angles associated with \(z_1\) and
  \(z_2\), respectively.
  Then \(\phi_1 = \arctan(1) = \frac{\pi}{4}\) and
  \(\phi_2 = \arctan(-1) = \frac{-\pi}{4}\).
  Then \(z_1 = \sqrt{2}e^{\pi i/4}\) and \(z_2 = \sqrt{2}e^{-\pi i/4}\).
  \begin{align*}
    z_1^n + z_2^n
    & = 2^{n/2}\bigl[e^{n\pi i/4} + e^{-n\pi i/4}\bigr]\\
    & = 2^{n/2 + 1}\biggl[\frac{e^{n\pi i/4} + e^{-n\pi i/4}}{2}\biggr]\\
    & = 2^{n/2 + 1}\cos\Bigl(\frac{n\pi}{4}\Bigr)
  \end{align*}
\item
  If \(z = x + iy\) (\(x\) and \(y\) real), find the real and imaginary parts
  of
  \[
  z^4, \qquad \frac{1}{z}, \qquad \frac{z - 1}{z + 1}, \qquad \frac{1}{z^2}
  \]
  For \(z^4\), we can use the binomial theorem since
  \((a + b)^n = \sum_{k = 0}^n\binom{n}{k}a^nb^{n - k}\).
  Therefore,
  \[
  (x + iy)^4 = \binom{4}{0}(iy)^4 + \binom{4}{1}x(iy)^3 +
  \binom{4}{2}x^2(iy)^2 + \binom{4}{3}x^3(iy) + \binom{4}{4}x^4 =
  y^4 - 4xy^3i - 6x^2y^2 + 4x^3yi + x^4
  \]
  Then the real and imaginary parts are
  \begin{align*}
    u(x, y) & = x^4 + y^4 - 6x^2y^2\\
    v(x, y) & = 4x^3y - 4xy^3
  \end{align*}
  For second problem, we need to multiple by the conjugate \(\bar{z}\).
  \[
  \frac{1}{x + iy}\frac{x - iy}{x - iy} = \frac{x - iy}{x^2 + y^2}
  \]
  so the real and imaginary parts are
  \begin{align*}
    u(x, y) & = \frac{x}{x^2 + y^2}\\
    v(x, y) & = \frac{-y}{x^2 + y^2}
  \end{align*}
  For the third problem, we have \(\frac{x - 1 + iy}{x + 1 - iy}\).
  Then \(\bar{z} = x + 1 + iy\).
  \[
  \frac{x - 1 + iy}{x + 1 - iy}\frac{x + 1 + iy}{x + 1 + iy} =
  \frac{x^2 - 1 + 2xyi}{(x + 1)^2 + y^2}
  \]
  Then real and imaginary parts are
  \begin{align*}
    u(x, y) & = \frac{x^2 - 1}{(x + 1)^2 + y^2}\\
    v(x, y) & = \frac{2xy}{(x + 1)^2 + y^2}
  \end{align*}
  For the last problem, we have
  \[
  \frac{1}{z^2} = \frac{x^2 - y^2 - 2xyi}{x^4 + 2x^2y^2 + y^4}
  \]
  so the real and imaginary parts are
  \begin{align*}
    u(x, y) & = \frac{x^2 - y^2}{x^4 + 2x^2y^2 + y^4}\\
    v(x, y) & = \frac{-2xy}{x^4 + 2x^2y^2 + y^4}
  \end{align*}
\item
  Show that \(\bigl(\frac{-1\pm i\sqrt{3}}{2}\bigr)^3 = 1\) and
  \(\bigl(\frac{\pm 1\pm i\sqrt{3}}{2}\bigr)^6 = 1\).
  \par\smallskip
  Both problems will can be handled easily by converting to polar form.
  Let \(z_1 = \frac{-1\pm i\sqrt{3}}{2}\).
  Then \(\lvert z_1\rvert = 1\).
  Let \(\phi_+\) be the angle for the positive \(z_1\) and \(\phi_-\) for the
  negative.
  Then \(\phi_+ = \arctan(-\sqrt{3}) = \frac{2\pi}{3}\) and
  \(\phi_- = \arctan(\sqrt{3}) = \frac{4\pi}{3}\).
  We can write \(z_{1+} = e^{2i\pi/3}\) and \(z_{1-} = e^{4i\pi/3}\).
  \begin{align*}
    z_{1+}^3 & = e^{2i\pi}\\
             & = 1\\
    z_{1-}^3 & = e^{4i\pi}\\
             & = 1
  \end{align*}
  Therefore, \(z_1^3 = 1\).
  For the second problem, \(\phi_{ij} = \pm\frac{\pi}{3}\) and
  \(\pm\frac{2\pi}{3}\) for \(i, j = +, -\) and the \(\lvert z_2\rvert = 1\).
  When we raise \(z\) to the sixth poewr, the argument becomes \(\pm 2\pi\) and
  \(\pm 4\pi\).
  \[
  e^{\pm 2i\pi} = e^{\pm 4i\pi} = z^6 = 1
  \]
\end{enumerate}

\subsection{Square Roots}

\begin{enumerate}
\item
  Compute
  \[
  \sqrt{i}, \qquad \sqrt{-i}, \qquad \sqrt{1 + i}, \qquad
  \sqrt{\frac{1 - i\sqrt{3}}{2}}
  \]
  For \(\sqrt{i}\), we are looking for \(x\) and \(y\) such that
  \begin{align}
    \sqrt{i} & = x + iy\notag\\
    i & = x^2 - y^2 + 2xyi\notag\\
    x^2 - y^2 & = 0\label{1.1.2.1a}\\
    2xy & = 1\label{1.1.2.1b}
  \end{align}
  From \cref{1.1.2.1a}, we see that \(x^2 = y^2\) or \(\pm x = \pm y\).
  Also, note that \(i\) is the upper half plane (UHP).
  That is, the angle is positive so \(x = y\) and \(2x^2 = 1\) from
  \cref{1.1.2.1a}.
  Therefore, \(\sqrt{i} = \frac{1}{\sqrt{2}}(1 + i)\).
  We also could have done this problem using the polar form of \(z\).
  Let \(z = i\).
  Then \(z = e^{i\pi/2}\) so \(\sqrt{z} = e^{i\pi/4}\) which is exactly what we
  obtained.
  For \(\sqrt{-i}\), let \(z = -i\).
  Then \(z\) in polar form is \(z = e^{-i\pi/2}\) so
  \(\sqrt{z} = e^{-i\pi/4} = \frac{1}{\sqrt{2}}(1 - i)\).
  For \(\sqrt{1 + i}\), let \(z = 1 + i\).
  Then \(z = \sqrt{2}e^{i\pi/4}\) so \(\sqrt{z} = 2^{1/4}e^{i\pi/8}\).
  Finally, for \(\sqrt{\frac{1 - i\sqrt{3}}{2}}\), let
  \(z = \frac{1 - i\sqrt{3}}{2}\).
  Then \(z = e^{-i\pi/3}\) so
  \(\sqrt{z} = e^{-i\pi/6} = \frac{1}{2}(\sqrt{3} - i)\).
\item
  Find the four values of \(\sqrt[4]{-1}\).
  \par\smallskip
  Let \(z = \sqrt[4]{-1}\) so \(z^4 = -1\).
  Let \(z = re^{i\theta}\) so \(r^4e^{4i\theta} = -1 = e^{i\pi(1 + 2k)}\).
    \begin{align*}
      r^4 & = 1\\
      \theta & = \frac{\pi}{4}(1 + 2k)
    \end{align*}
    where \(k = 0\), \(1\), \(2\), \(3\).
    Since when \(k = 4\), we have \(k = 0\).
    Then \(\theta = \frac{\pi}{4}\), \(\frac{3\pi}{4}\), \(\frac{5\pi}{4}\),
    and \(\frac{7\pi}{4}\).
    \[
    z = e^{i\pi/4}, e^{3i\pi/4}, e^{5i\pi/4}, e^{7i\pi/4}
    \]
\item
  Compute \(\sqrt[4]{i}\) and \(\sqrt[4]{-i}\).
  \par\smallskip
  Let \(z = \sqrt[4]{i}\) and \(z = re^{i\theta}\).
  Then \(r^4e^{4i\theta} = i = e^{i\pi/2}\).
    \begin{align*}
      r^4 & = 1\\
      \theta & = \frac{\pi}{8}
    \end{align*}
    so \(z = e^{i\pi/8}\).
    Now, let \(z = \sqrt[4]{-i}\).
    Then \(r^4e^{4i\theta} = e^{-i\pi/2}\) so \(z = e^{-i\pi/8}\).
\item
  Solve the quadratic equation
  \[
  z^2 + (\alpha + i\beta)z + \gamma + i\delta = 0.
  \]
  The quadratic equation is \(x = \frac{-b\pm\sqrt{b^2 - ac}}{2}\).
  For the complex polynomial, we have
  \[
  z = \frac{-\alpha - \beta i\pm
    \sqrt{\alpha^2 - \beta^2 - 4\gamma + i(2\alpha\beta - 4\delta)}}{2}
  \]
  Let
  \(a + bi = \sqrt{\alpha^2 - \beta^2 - 4\gamma + i(2\alpha\beta - 4\delta)}\).
  Then
  \[
  z = \frac{-\alpha - \beta\pm (a + bi)}{2}
  \]
\end{enumerate}

\subsection{Justification}

\begin{enumerate}
\item
  Show that the system of all matrices of the special form
  \[
  \begin{pmatrix}
    \alpha & \beta\\
    -\beta & \alpha
  \end{pmatrix},
  \]
  combined by matrix addition and matrix multiplication, is isomorphic to the
  field of complex numbers.
\item
  Show that the complex number system can be thought of as the field of all
  polynomials with real coefficients modulo the irreducible polynomial
  \(x^2 + 1\).
\end{enumerate}

\subsection{Conjugation, Absolute Value}

\begin{enumerate}
\item
  Verify by calculation the values of
  \[
  \frac{z}{z^2 + 1}
  \]
  for \(z = x + iy\) and \(\bar{z} = x - iy\) are conjugate.
  \par\smallskip
  For \(z\), we have that \(z^2 = x^2 - y^2 + 2xyi\).
  \begin{align*}
    \frac{z}{z^2 + 1}
    & = \frac{x + iy}{x^2 - y^2 + 1 + 2xyi}\\
    & = \frac{x + iy}{x^2 - y^2 + 1 + 2xyi}
      \frac{x^2 - y^2 + 1 - 2xyi}{x^2 - y^2 + 1 - 2xyi}\\
    & = \frac{x(x^2 - y^2 + 1) + 2xy^2 + iy(x^2 - y^2 + 1 - 2x^2)}
      {(x^2 - y^2 + 1)^2 + 4x^2y^2}\eqnumtag\label{1.1.4.1z}
  \end{align*}
  For \(\bar{z}\), we have that \(\bar{z}^2 = x^2 - y^2 - 2xyi\).
  \begin{align*}
    \frac{\bar{z}}{\bar{z}^2 + 1}
    & = \frac{x - iy}{x^2 - y^2 + 1 - 2xyi}\\
    & = \frac{x - iy}{x^2 - y^2 + 1 - 2xyi}
      \frac{x^2 - y^2 + 1 + 2xyi}{x^2 - y^2 + 1 + 2xyi}\\
    & = \frac{x(x^2 - y^2 + 1) + 2xy^2 - iy(x^2 - y^2 + 1 - 2x^2)}
      {(x^2 - y^2 + 1)^2 + 4x^2y^2}\eqnumtag\label{1.1.4.1barz}
  \end{align*}
  Therefore, we have that \cref{1.1.4.1z,1.1.4.1barz} are conjugates.
\item
  Find the absolute value (modulus) of
  \[
  -2i(3 + i)(2 + 4i)(1 + i)\qquad\text{and}\qquad
  \frac{(3 + 4i)(-1 + 2i)}{(-1 - i)(3 - i)}.
  \]
  When we expand the first problem, we have that
  \[
  z_1 = -2i(3 + i)(2 + 4i)(1 + i) = 32 + 24i
  \]
  so
  \[
  \lvert z_1\rvert = \sqrt{32^2 + 24^2} = 40.
  \]
  For the second problem, we have that
  \[
  z_2 = \frac{(3 + 4i)(-1 + 2i)}{(-1 - i)(3 - i)} = 2 - \frac{3}{2}i
  \]
  so
  \[
  \lvert z_2\rvert = \sqrt{4 + 9/4} = \frac{5}{2}.
  \]
\item
  Prove that
  \[
  \Bigl\lvert\frac{a - b}{1 - \bar{a}b}\Bigr\rvert = 1
  \]
  if either \(\lvert a\rvert = 1\) or \(\lvert b\rvert = 1\).
  What exception must be made if \(\lvert a\rvert = \lvert b\rvert = 1\)?
  \par\smallskip
  Recall that \(\lvert z\rvert^2 = z\bar{z}\).
  \begin{align*}
    1^2 & = \Bigl\lvert\frac{a - b}{1 - \bar{a}b}\Bigr\rvert^2\\
    1 & = \Bigl(\frac{a - b}{1 - \bar{a}b}\Bigr)
    \Bigl(\frac{\overline{a - b}}{\overline{1 - \bar{a}b}}\Bigr)\\
        & = \Bigl(\frac{a - b}{1 - \bar{a}b}\Bigr)
          \Bigl(\frac{\bar{a} - \bar{b}}{1 - a\bar{b}}\Bigr)\\
        & = \frac{a\bar{a} - a\bar{b} - \bar{a}b + b\bar{b}}
          {1 - \bar{a}b - a\bar{b} + a\bar{a}b\bar{b}}\eqnumtag\label{1.1.4.3}
  \end{align*}
  If \(\lvert a\rvert = 1\), then \(\lvert a\rvert^2 = a\bar{a} = 1\) and
  similarly for \(\lvert b\rvert^2 = 1\).
  Then \cref{1.1.4.3} becomes
  \[
  \frac{1 - a\bar{b} - \bar{a}b + b\bar{b}}{1 - \bar{a}b - a\bar{b} + b\bar{b}}
  \qquad\text{and}\qquad
  \frac{1 - a\bar{b} - \bar{a}b + a\bar{a}}{1 - \bar{a}b - a\bar{b} + a\bar{a}}
  \]
  resepctively which is one.
  If \(\lvert a\rvert = \lvert b\rvert = 1\), then
  \(\lvert a\rvert^2 = \vert b\rvert^2 = 1\) so \cref{1.1.4.3} can be written as
  \[
  \frac{2 - a\bar{b} - \bar{a}b}{2 - \bar{a}b - a\bar{b}}.
  \]
  Therefore, we must have that \(a\bar{b} + \bar{a}b\neq 2\).
\item
  Find the conditions under which the equation \(az + b\bar{z} + c = 0\) in one
  complex unknown has exactly one solution, and compute that solution.
  \par\smallskip
  Let \(z = x + iy\).
  Then \(az + b\bar{z} + c = a(x + iy) + b(x - iy) + c = 0\).
  \begin{subequations}
    \begin{align}
      (a + b)x + c & = 0\label{1.1.4.4R}\\
      (a - b)y & = 0\label{1.1.4.4I}
    \end{align}
  \end{subequations}
  Lets consider \cref{1.1.4.4I}.
  We either have that \(a = b\) or \(y = 0\).
  If \(a = b\), then WLOG \cref{1.1.4.4R} can be written as
  \[
  x = \frac{-c}{2a}
  \]
  and \(y\in\mathbb{R}\).
  For fixed \(a,b,c\), we have infinitely many solutions when \(a = b\) since
  \(z = \frac{-c}{2a} + iy\) for \(y\in\mathbb{R}\).
  If \(y = 0\), then \cref{1.1.4.4R} can be written as
  \[
  x = \frac{-c}{a + b}.
  \]
  Therefore, \(z = x\) and we have only one solution.
\item
  Prove that Lagrange's identity in the complex form
  \[
  \Bigl\lvert\sum_{i = 1}^na_ib_i\Bigr\rvert^2 =
  \sum_{i = 1}^n\lvert a_i\rvert^2\sum_{i = 1}^n\lvert b_i\rvert^2 -
  \smashoperator{\sum_{1\leq i\leq j\leq n}}
  \lvert a_i\bar{b}_j - a_j\bar{b}_i\rvert^2.
  \]
\end{enumerate}

\subsection{Inequalities}

\begin{enumerate}
\item
  Prove that
  \[
  \Bigl\lvert\frac{a - b}{1 - \bar{a}b}\Bigr\rvert < 1
  \]
  if \(\lvert a\rvert < 1\) and \(\lvert b\rvert < 1\).
  \par\smallskip
  From the properties of the modulus, we have that
  \begin{align*}
    \Bigl\lvert\frac{a - b}{1 - \bar{a}b}\Bigr\rvert
    & = \frac{\lvert a - b\rvert}{\lvert 1 - \bar{a}b\rvert}\\
    & = \frac{\lvert a - b\rvert^2}{\lvert 1 - \bar{a}b\rvert^2}\eqnumtag
    \label{1.1.5.1ineq1}\\
    & = \frac{(a - b)(\bar{a} - \bar{b})}{(1 - \bar{a}b)(1 - a\bar{b})}\\
    & = \frac{\lvert a\rvert^2 + \lvert b\rvert^2 - a\bar{b} - \bar{a}b}
      {1 + \lvert a\rvert^2\lvert b\rvert^2 - \bar{a}b - a\bar{b}}\\
    & < \frac{2 - a\bar{b} - \bar{a}b}{2 - \bar{a}b - a\bar{b}}\\
    & = 1\eqnumtag\label{1.1.5.1ineq2}
  \end{align*}
  From \cref{1.1.5.1ineq1,1.1.5.1ineq2}, we have
  \begin{align*}
    \frac{\lvert a - b\rvert^2}{\lvert 1 - \bar{a}b\rvert^2} & < 1\\
    \frac{\lvert a - b\rvert}{\lvert 1 - \bar{a}b\rvert} & < 1
  \end{align*}
\item
  Prove Cauchy's inequality by induction.
  \par\smallskip
  Cauchy's inequality is
  \begin{align*}
    \lvert a_1b_1 + \cdots + a_nb_n\rvert^2
    & \leq \bigl(\lvert a_1\rvert^2 + \cdots + \lvert a_n\rvert^2\bigr)
      \bigl(\lvert b_1\rvert^2 + \cdots + \lvert b_n\rvert^2\bigr)\\
    \intertext{which can be written more compactly as}
    \Bigl\lvert\sum_{i = 1}^na_ib_i\Bigr\rvert^2
    & \leq\sum_{i = 1}^n\lvert a_i\rvert^2\sum_{i = 1}^n\lvert b_i\rvert^2.
  \end{align*}
  For the base case, \(i = 1\), we have
  \[
  \lvert a_1b_1\rvert^2 = (a_1b_1)(\bar{a}_1\bar{b}_1) =
  a_1\bar{a}_1b_1\bar{b}_1 = \lvert a_1\rvert^2\lvert b_1\rvert^2
  \]
  so the base case is true.
  Now let the equality hold for all \(k - 1\in\mathbb{Z}\) where
  \(k - 1\leq n\).
  That is, we assume that
  \[
  \Bigl\lvert\sum_{i = 1}^{k - 1}a_ib_i\Bigr\rvert^2\leq
  \sum_{i = 1}^{k - 1}\lvert a_i\rvert^2\sum_{i = 1}^{k - 1}\lvert b_i\rvert^2
  \]
  to be true.
  \begin{align*}
    \Bigl\lvert\sum_{i = 1}^{k - 1}a_ib_i\Bigr\rvert^2 + \lvert a_kb_k\rvert^2
    & \leq \sum_{i = 1}^{k - 1}\lvert a_i\rvert^2
      \sum_{i = 1}^{k - 1}\lvert b_i\rvert^2 + \lvert a_kb_k\rvert^2\\
    \Bigl\lvert\sum_{i = 1}^ka_ib_i\Bigr\rvert^2
    & \leq \sum_{i = 1}^{k - 1}\lvert a_i\rvert^2
      \sum_{i = 1}^{k - 1}\lvert b_i\rvert^2 + (a_kb_k)(\bar{a}_k\bar{b}_k)\\
    & = \sum_{i = 1}^{k - 1}\lvert a_i\rvert^2
      \sum_{i = 1}^{k - 1}\lvert b_i\rvert^2 + \lvert a_k\rvert^2
      \lvert b_k\rvert^2\\
    & = \sum_{i = 1}^k\lvert a_i\rvert^2\sum_{i = 1}^k\lvert b_i\rvert^2
  \end{align*}
  Therefore, by the principal of mathematical induction, Cauchy's inequality is
  true for all \(n\geq 1\) for \(n\in\mathbb{Z}^+\).
\item
  If \(\lvert a_i\rvert < 1\), \(\lambda_i\geq 0\) for \(i = 1,\ldots,n\) and
  \(\lambda_1 + \lambda_2 + \cdots + \lambda_n = 1\), show that
  \[
  \lvert\lambda_1a_1 + \lambda_2a_2 + \cdots + \lambda_na_n\rvert < 1.
  \]
  Since \(\sum_{i = 1}^n\lambda_i = 1\) and \(\lambda_i\geq 0\),
  \(0\leq \lambda_i < 1\).
  By the triangle inequality,
  \begin{align*}
    \lvert\lambda_1a_1 + \lambda_2a_2 + \cdots + \lambda_na_n\rvert
    & \leq\lvert\lambda_1\rvert\lvert a_1\rvert + \cdots +
      \lvert a_n\rvert\lvert\lambda_n\rvert\\
    & < \sum_{i = 1}^n\lambda_i\\
    & = 1
  \end{align*}
\item
  Show that there are complex numbers \(z\) satisfying
  \[
  \lvert z - a\rvert + \lvert z + a\rvert = 2\lvert c\rvert
  \]
  if and only if \(\lvert a\rvert\leq\lvert c\rvert\).
  If this condition is fulfilled, what are the smallest and largest values
  \(\lvert z\rvert\)?
\end{enumerate}

\section{The Geometric Representation of Complex Numbers}

\subsection{Geometric Addition and Multiplication}

\begin{enumerate}
\item
  Find the symmetric points of \(a\) with respect to the lines which bisect the
  angles between the coordinate axes.
\item
  Prove that the points \(a_1,a_2,a_3\) are vertices of an equilateral triangle
  if and only if \(a_1^2 + a_2^2 + a_3^2 = a_1a_2 + a_2a_3 + a_1a_3\).
\item
  Suppose that \(a\) and \(b\) are two vertices of a square.
  Find the two other vertices in all possible cases.
\item
  Find the center and the radius of the circle which circumscribes the triangle
  with vertices \(a_1,a_2,a_3\).
  Express the result in symmetric form.
\end{enumerate}

\subsection{The Binomial Equation}

\begin{enumerate}
\item
  Express \(\cos(3\varphi)\), \(\cos(4\varphi)\), and \(\sin(5\varphi)\) in
  terms of \(\cos(\varphi)\) and \(\sin(\varphi)\).
  \par\smallskip
  For these problems, the sum addition identities will be employed; that is,
  \begin{align*}
    \cos(\alpha\pm\beta)
    & = \cos(\alpha)\cos(\beta)\mp\sin(\alpha)\sin(\beta)\\
    \sin(\alpha\pm\beta)
    & = \sin(\alpha)\cos(\beta)\pm\sin(\beta)\cos(\alpha)
  \end{align*}
  We can write \(\cos(3\varphi)\) as \(\cos(2\varphi + \varphi)\) so
  \begin{align*}
    \cos(3\varphi)
    & = \cos(2\varphi + \varphi)\\
    & = \cos(2\varphi)\cos(\varphi) - \sin(2\varphi)\sin(\varphi)\\
    & = \bigl[\cos^2(\varphi) - \sin^2(\varphi)\bigr]\cos(\varphi) -
      2\sin(\varphi)\cos(\varphi)\sin(\varphi)\\
    & = \cos^3(\varphi) - 3\sin^2(\varphi)\cos(\varphi)
  \end{align*}
  For \(\cos(4\varphi)\), we have
  \begin{align*}
    \cos(4\varphi)
    & = \cos(2\varphi)\cos(2\varphi) - \sin(2\varphi)\sin(2\varphi)\\
    & = \bigl[\cos^2(\varphi) - \sin^2(\varphi)\bigr]^2 -
      4\sin^2(\varphi)\cos^2(\varphi)\\
    & = \cos^4(\varphi) + \sin^4(\varphi) - 6\sin^2(\varphi)\cos^2(\varphi)
  \end{align*}
  For \(\sin(5\varphi)\), we have
  \begin{align*}
    \sin(5\varphi)
    & = \sin(4\varphi)\cos(\varphi) + \sin(\varphi)\cos(4\varphi)\\
    & = 2\sin(2\varphi)\cos(2\varphi)
      \bigl[\cos^2(\varphi) - \sin^2(\varphi)\bigr]\cos(\varphi) +
      \sin^5(\varphi) + \sin(\varphi)\cos^4(\varphi) -
      6\sin^3(\varphi)\cos^2(\varphi)\\
    & = 5\sin(\varphi)\cos^4(\varphi) - 10\sin^3(\varphi)\cos^2(\varphi) +
      \sin^5(\varphi)
  \end{align*}
\item
  Simplify \(1 + \cos(\varphi) + \cos(2\varphi) + \cdots + \cos(n\varphi)\) and
  \(\sin(\varphi) + \cdots + \sin(n\varphi)\).
  \par\smallskip
  Instead of considering the two separate series, we will consider the series
  \begin{align*}
    1 + \cos(\varphi) + i\sin(\varphi) + \cdots + \cos(n\varphi) +
    i\sin(n\varphi)
    & = 1 + e^{i\varphi} + e^{2i\varphi} + \cdots + e^{ni\varphi}\\
    & = \sum_{k = 0}^ne^{ki\varphi}\\
    \intertext{Recall that
    \(\sum_{k = 0}^{n - 1}r^k = \frac{1 - r^k}{1 - r}\). So}
    & = \frac{1 - e^{i\varphi(n + 1)}}{1 - e^{i\varphi}}\\
    & = \frac{e^{i\varphi(n + 1)} - 1}{e^{i\varphi} - 1}
      \eqnumtag\label{1.2.2.2}
  \end{align*}
  Note that
  \(\sin(\frac{\theta}{2}) = \frac{e^{i\theta/2} - e^{-i\theta/2}}{2i}\) so
  \(2ie^{i\theta/2}\sin(\frac{\theta}{2}) = e^{i\theta} - 1\).
  We can now write \cref{1.2.2.2} as
  \begin{align*}
    \sum_{k = 0}^ne^{ki\varphi}
    & = \frac{e^{i\varphi(n + 1)/2}\sin\bigl(\frac{\varphi(n + 1)}{2}\bigr)}
      {e^{i\varphi/2}\sin\bigl(\frac{\varphi}{2}\bigr)}\\
    & = \frac{\sin\bigl(\frac{\varphi(n + 1)}{2}\bigr)}
      {\sin\bigl(\frac{\varphi}{2}\bigr)}e^{in\varphi/2}
      \eqnumtag\label{1.2.2.2sum}
  \end{align*}
  By taking the real and imaginary parts of \cref{1.2.2.2sum}, we get the
  series for \(\sum_{k = 0}^n\cos(n\varphi)\) and
  \(\sum_{k = 0}^n\sin(n\varphi)\), respectively.
  \begin{align*}
    \sum_{k = 0}^n\cos(n\varphi)
    & = \frac{\sin\bigl(\frac{\varphi(n + 1)}{2}\bigr)}
      {\sin\bigl(\frac{\varphi}{2}\bigr)}\cos\Bigl(\frac{n\varphi}{2}\Bigr)\\
    \sum_{k = 0}^n\sin(n\varphi)
    & = \frac{\sin\bigl(\frac{\varphi(n + 1)}{2}\bigr)}
      {\sin\bigl(\frac{\varphi}{2}\bigr)}\sin\Bigl(\frac{n\varphi}{2}\Bigr)
  \end{align*}
\item
  Express the fifth and tenth roots of unity in algebraic form.
  \par\smallskip
  To find the roots of unity, we are looking to solve \(z^n = 1\).
  Let \(z = e^{i\theta}\) and \(1 = e^{2ik\pi}\).
  Then \(\theta = \frac{2k\pi}{n}\).
  For the fifth roots of unity, \(n = 5\) and \(k = 0,1,\ldots, 4\) so we have
  \begin{alignat*}{2}
    \omega_0 & = e^0 &&{}= \cos(0) + i\sin(0)\\
    & &&{}= 1\\
    \omega_1 & = e^{2\pi/5} &&{}=
    \cos\Bigl(\frac{2\pi}{5}\Bigr) + i\sin\Bigl(\frac{2\pi}{5}\Bigr)\\
    \omega_2 & = e^{4\pi/5} &&{}=
    \cos\Bigl(\frac{4\pi}{5}\Bigr) + i\sin\Bigl(\frac{4\pi}{5}\Bigr)\\
    \omega_3 & = e^{6\pi/5} &&{}=
    \cos\Bigl(\frac{6\pi}{5}\Bigr) + i\sin\Bigl(\frac{6\pi}{5}\Bigr)\\
    \omega_4 & = e^{8\pi/5} &&{}=
    \cos\Bigl(\frac{8\pi}{5}\Bigr) + i\sin\Bigl(\frac{8\pi}{5}\Bigr)\\
  \end{alignat*}
  Now we can plot the roots of unity on the unit circle.
  \begin{figure}[H]
    \centering
    \includestandalone[mode = image, width = 2in]{Tikz/1-2prob3a}
    \caption{The fifth roots of unity.}
  \end{figure}
  For the tenth roots of unity, \(n = 10\) and \(k = 0,1,\ldots, 9\) so we have
  \begin{alignat*}{2}
    \omega_0 & = e^0 &&{}= \cos(0) + i\sin(0)\\
    & &&{}= 1\\
    \omega_1 & = e^{2\pi/10} &&{}=
    \cos\Bigl(\frac{\pi}{5}\Bigr) + i\sin\Bigl(\frac{\pi}{5}\Bigr)\\
    \omega_2 & = e^{4\pi/10} &&{}=
    \cos\Bigl(\frac{2\pi}{5}\Bigr) + i\sin\Bigl(\frac{2\pi}{5}\Bigr)\\
    \omega_3 & = e^{6\pi/10} &&{}=
    \cos\Bigl(\frac{3\pi}{5}\Bigr) + i\sin\Bigl(\frac{3\pi}{5}\Bigr)\\
    \omega_4 & = e^{8\pi/10} &&{}=
    \cos\Bigl(\frac{4\pi}{5}\Bigr) + i\sin\Bigl(\frac{4\pi}{5}\Bigr)\\
    \omega_5 & = e^{10\pi/10} &&{}= \cos(\pi) + i\sin(\pi)\\
    & &&{}= -1\\
    \omega_6 & = e^{12\pi/10} &&{}=
    \cos\Bigl(\frac{6\pi}{5}\Bigr) + i\sin\Bigl(\frac{6\pi}{5}\Bigr)\\
    \omega_7 & = e^{14\pi/10} &&{}=
    \cos\Bigl(\frac{7\pi}{5}\Bigr) + i\sin\Bigl(\frac{7\pi}{5}\Bigr)\\
    \omega_8 & = e^{16\pi/10} &&{}=
    \cos\Bigl(\frac{8\pi}{5}\Bigr) + i\sin\Bigl(\frac{8\pi}{5}\Bigr)\\
    \omega_9 & = e^{18\pi/10} &&{}=
    \cos\Bigl(\frac{9\pi}{5}\Bigr) + i\sin\Bigl(\frac{9\pi}{5}\Bigr)
  \end{alignat*}
  Now we can plot the roots of unity on the unit circle.
  \begin{figure}[H]
    \centering
    \includestandalone[mode = image, width = 2in]{Tikz/1-2prob3b}
    \caption{The tenth roots of unity.}
  \end{figure}
\item
  If \(\omega\) is given by \(\omega = \cos\bigl(\frac{2\pi}{n}\bigr) +
  i\sin\bigl(\frac{2\pi}{n}\bigr)\), prove that
  \[
  1 + \omega^h + \omega^{2h} + \cdots + \omega^{(n - 1)h} = 0
  \]
  for any integer \(h\) which is not a multiple of \(n\).
  \par\smallskip
  Let \(\omega = \cos\bigl(\frac{2\pi}{n}\bigr) +
  i\sin\bigl(\frac{2\pi}{n}\bigr)\) be written in exonential form as
  \(\omega = e^{2\pi i/n}\).
  Then the series can be written as
  \[
  \sum_{k = 0}^{n - 1}\bigl(e^{2\pi ih/n}\bigr)^k =
  \frac{e^{2ih\pi} - 1}{e^{2hi\pi/n} - 1}.
  \]
  Since \(h\) is an integer, \(e^{2ih\pi} = 1\); therefore, the series zero.
\item
  What is the value of
  \[
  1 - \omega^h + \omega^{2h} - \cdots + (-1)^{n - 1}\omega^{(n - 1)h}\mbox{?}
  \]
  We can represent this series similarly as
  \[
  \sum_{k = 0}^{n - 1}\bigl(-e^{2\pi ih/n}\bigr)^k =
  \frac{(-1)^ne^{2ih\pi} - 1}{-e^{2hi\pi/n} - 1} =
  \frac{1 + (-1)^{n + 1}e^{2ih\pi}}{1 + e^{2hi\pi/n}}.
  \]
  Again, since \(h\) is an intger, we have that \(e^{2ih\pi} = 1\) which leaves
  us with
  \[
  \frac{1 + (-1)^{n + 1}}{1 + e^{2hi\pi/n}} =
  \begin{cases}
    0, & \text{if \(n\) is even}\\
    \frac{2}{1 + e^{2hi\pi/n}}, & \text{if \(n\) is odd}
  \end{cases}
  \]
\end{enumerate}

\subsection{Analytic Geometry}

\subsection{The Spherical Representation}

%%% Local Variables:
%%% mode: latex
%%% TeX-master: t
%%% End:
