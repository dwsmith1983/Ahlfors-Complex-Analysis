\chapter{Complex Functions}

\section{Introduction to the Concept of Analytical Function}

\subsection{Limits and Continuity}

\subsection{Analytic Functions}

\begin{enumerate}
\item
  If \(g(w)\) and \(f(z)\) are analytic functions, show that \(g(f(z))\) is
  also analytic.
  \par\smallskip
  Let \(g(w) = h(x,y) + it(x,y)\) and \(f(z) = u(x,y) + iv(x,y)\) where
  \(z = w = x + iy\) for \(x,y\in\mathbb{R}\).
  Then
  \[
  (g\circ f)(z) = h(u(x, y), v(x, y)) + it(u(x, y), v(x, y)).
  \]
  Since \(f\) and \(g\) satisfy the Cauchy-Riemann equations,
  \begin{alignat*}{4}
    \frac{\partial u}{\partial x} & = \frac{\partial v}{\partial y} & \qquad &
    \frac{\partial u}{\partial y} && ={} -\frac{\partial v}{\partial x}\\
    \frac{\partial h}{\partial x} & = \frac{\partial t}{\partial y} & \qquad &
    \frac{\partial h}{\partial u} && ={} -\frac{\partial t}{\partial x}
  \end{alignat*}
  The partial deivatives of \((g\circ f)(z)\) are
  \begin{alignat*}{4}
    \frac{\partial h}{\partial x}
    & = \frac{\partial h}{\partial u}\frac{\partial u}{\partial x} +
    \frac{\partial t}{\partial v}\frac{\partial v}{\partial x} & \qquad &
    \frac{\partial t}{\partial y}
    && ={} \frac{\partial t}{\partial u}\frac{\partial u}{\partial y} +
    \frac{\partial t}{\partial v}\frac{\partial v}{\partial y}\\
    \frac{\partial h}{\partial y}
    & = \frac{\partial h}{\partial u}\frac{\partial u}{\partial y} +
    \frac{\partial t}{\partial v}\frac{\partial v}{\partial y} & \qquad &
    \frac{\partial t}{\partial x}
    && ={} \frac{\partial t}{\partial u}\frac{\partial u}{\partial x} +
    \frac{\partial t}{\partial v}\frac{\partial v}{\partial x}
  \end{alignat*}
  In order for \(g(f(z))\) to be analytic,
  \(\frac{\partial h}{\partial x} = \frac{\partial t}{\partial y}\) and
  \(\frac{\partial h}{\partial y} = -\frac{\partial t}{\partial x}\).
  We can then write
  \begin{align*}
    \frac{\partial h}{\partial x} - \frac{\partial t}{\partial y}
    & = \frac{\partial h}{\partial u}\frac{\partial u}{\partial x} +
      \frac{\partial h}{\partial v}\frac{\partial v}{\partial x} -
      \frac{\partial t}{\partial u}\frac{\partial u}{\partial y} -
      \frac{\partial t}{\partial v}\frac{\partial v}{\partial y}\\
    & = \underbrace{\frac{\partial h}{\partial u}\frac{\partial u}{\partial x}
      - \frac{\partial t}{\partial u}\frac{\partial u}{\partial y}}_{
      \text{term } 1} +
      \underbrace{\frac{\partial h}{\partial v}\frac{\partial v}{\partial x} -
      \frac{\partial t}{\partial v}\frac{\partial v}{\partial y}}_{\text{term }
      2}\eqnumtag\label{2.1.2.1crx}\\
    \intertext{In order for the right hand side of \cref{2.1.2.1crx} to be
    zero, we need both terms to be zero.}
    \frac{\partial h}{\partial u}\frac{\partial u}{\partial x} -
    \frac{\partial t}{\partial u}\frac{\partial u}{\partial y}
    & = \frac{\partial h}{\partial u}\frac{\partial u}{\partial x} -
      \frac{\partial t}{\partial y}\frac{\partial u}{\partial u}\\
    & = \frac{\partial h}{\partial u}\frac{\partial u}{\partial x} -
      \frac{\partial h}{\partial u}\frac{\partial u}{\partial x}\eqnumtag
      \label{2.1.2.1ux}\\
    \intertext{\Cref{2.1.2.1ux} occurs since \(g\) is analytic and satisfies
    the Cauchy-Riemann equations.}
    & = 0\\
    \intertext{For the second term in \cref{2.1.2.1crx}, we again use the
    analyticity of \(g\).}
    \frac{\partial h}{\partial v}\frac{\partial v}{\partial x} -
    \frac{\partial t}{\partial v}\frac{\partial v}{\partial y}
    & = \frac{\partial h}{\partial v}\frac{\partial v}{\partial x} -
      \frac{\partial h}{\partial v}\frac{\partial v}{\partial x}\\
    & = 0\\
    \intertext{Therefore, from \cref{2.1.2.1crx}, we have}
    \frac{\partial h}{\partial x} - \frac{\partial t}{\partial y} & = 0\\
    \frac{\partial h}{\partial x} & = \frac{\partial t}{\partial y}
  \end{align*}
  By similar analysis, we are able to conclude that
  \(\frac{\partial h}{\partial y} = -\frac{\partial t}{\partial x}\).
  Therefore, \(g(f(z))\) satisfies the Cauchy-Riemann so it is analytic.
\item
  Verify Cauchy-Riemann's equations for the function \(z^2\) and \(z^3\).
  \par\smallskip
  Let \(z = x + iy\).
  Then \(z^2 = x^2 - y^2 + 2xyi\) and
  \(z^3 = x^3 - 3xy^2 + i(3x^2y - y^3)\).
  For \(f(z) = z^2\), the Cauchy-Riemann equations are
  \begin{gather*}
    u_x = 2x \qquad v_y = 2x\\
    u_y = -2y \qquad -v_x = -2y
  \end{gather*}
  Thus, the Cauchy-Riemann equation satisfied for \(f(z) = z^2\).
  For \(f(z) = z^3\), the Cauchy-Riemann equations are
  \begin{gather*}
    u_x = 3x^2 - 3y^2 \qquad v_y = 3x^2 - 3y^2\\
    u_y = -6xy \qquad -v_x = -6xy
  \end{gather*}
  Thus, the Cauchy-Riemann equation satisfied for \(f(z) = z^3\).
\item
  Find the most general harmonic polynomial of the form
  \(ax^3 + bx^2y + cxy^2 + dy^3\).
  Determine the conjugate harmonic function and the corresponding analytic
  function by integration and by the formal method.
  \par\smallskip
  In order to be harmonic, \(u(x,y) = ax^3 + bx^2y + cxy^2 + dy^3\) has to
  satisfy \(\nabla^2u = 0\) so
  \[
  u_{xx} + u_{yy} = (3a + c)x + (3d + b)y = 0.
  \]
  Thus, \(3a = -c\) and \(3d = -b\) so
  \[
  u(x,y) = ax^3 - 3axy^2 - 3dx^2y + dy^3.
  \]
  To find the harmonic conjugate \(v(x,y\), we need to look at the
  Cauchy-Riemann equations.
  By the Cauchy-Riemann equations,
  \[
  u_x = 3ax^2 - 3ay^2 - 6dxy = v_y.
  \]
  Then we can integrate with respect to \(y\) to find \(v(x,y)\).
  \[
  v(x,y) = \int(3ax^2 - 3ay^2 - 6dxy)dy = 3ax^2y - ay^3 - 3dxy^2 + g(x)
  \]
  Using the second Cauchy-Riemann, we have
  \[
  v_x = 6axy - 3dy^2 + g'(x) = -u_y = 3dx^2 + 6axy - 3dy^2
  \]
  so \(g'(x) = 3dx^2\).
  Then \(g(x) = dx^3 + C\) and
  \[
  v(x,y) = 3ax^2y - ay^3 - 3dxy^2 + dx^3 + C.
  \]
\item
  Show that an analytic function cannot have a constant absolute value without
  reducing to a constant.
  \par\smallskip
  Let \(f = u(x,y) + iv(x,y)\).
  Then the modulus of \(f\) is \(\lvert f\rvert = \sqrt{u^2 + v^2}\).
  If the modulus of \(f\) is constant, then \(u^2 + v^2 = c\) for some constant
  \(c\).
  If \(c = 0\), then \(f = 0\) which is constant.
  Suppose \(c\neq 0\).
  By taking the derivative with respect to \(x\) and \(y\), we have
  \begin{align*}
    0 & = \frac{\partial}{\partial x}(u^2 + v^2)\\
      & = 2uu_x + 2vv_x\\
      & = uu_x + vv_x\\
    0 & = \frac{\partial}{\partial y}(u^2 + v^2)\\
      & = uu_y + vv_y
  \end{align*}
  Since \(f\) is analytic, \(f\) satisfies the Cauchy-Riemann.
  That is, \(u_x = v_y\) and \(u_y = -v_x\).
  \begin{subequations}
    \begin{align}
      uu_x - vu_y & = 0\label{2.1.2.4a}\\
      uu_y + vu_x & = 0\label{2.1.2.4b}
    \end{align}
  \end{subequations}
  Let's write \cref{2.1.2.4a,2.1.2.4b} in matrix form.
  Then we have
  \[
  \begin{bmatrix}
    u & -v\\
    v & u
  \end{bmatrix}
  \begin{bmatrix}
    \frac{\partial u}{\partial x}\\
    \frac{\partial u}{\partial x}
  \end{bmatrix} =
  \begin{bmatrix}
    0\\
    0
  \end{bmatrix}
  \]
  Suppose the matrix is not invertible.
  Then \(u^2 + v^2 = 0\).
  Since \(u^2,v^2\in\mathbb{R}\), \(u^2,v^2\geq 0\).
  Therefore, \(u = v = 0\) so \(f(z) = 0\).
  Now, suppose that the matrix is invertible.
  Then we have
  \[
  \begin{bmatrix}
    \frac{\partial u}{\partial x}\\
    \frac{\partial u}{\partial x}
  \end{bmatrix} =
  \begin{bmatrix}
    0\\
    0
  \end{bmatrix}
  \]
  so \(f'(z) = 0\) and \(f(z) = c\) for some constant \(c\).
\item
  Prove rigorously that the functions \(f(z)\) and \(\overline{f(\bar{z})}\)
  are simultaneously analytic.
  \par\smallskip
  Let \(g(z) = \overline{f(\bar{z})}\) and suppose \(f\) is analytic.
  Then \(g'(z)\)  is
  \begin{align*}
    g'(z) & = \lim_{\Delta z\to 0}\frac{g(z + \Delta z) - g(z)}{\Delta z}\\
          & = \lim_{\Delta z\to 0}
            \frac{\overline{f(\bar{z} + \overline{\Delta z})} -
            \overline{f(\bar{z})}}{\Delta z}\\
          & = \lim_{\Delta z\to 0}\biggl[
            \overline{\frac{f(\bar{z} + \overline{\Delta z}) - f(\bar{z})}
            {\overline{\Delta z}}}\biggr]\\
    \intertext{Since conjugation is continuous, we can move the limit inside
    the conjugation.}
          & = \overline{\lim_{\Delta z\to 0}
            \frac{f(\bar{z} + \overline{\Delta z}) - f(\bar{z})}
            {\overline{\Delta z}}}\\
          & = \overline{f'(\bar{z})}
  \end{align*}
  Thus, \(g\) is differentiable with derivative \(\overline{f'(\bar{z})}\).
  Suppose \(\overline{f(\bar{z})}\) is analytic and let
  \(\overline{g(\bar{z})} = f(z)\).
  Then by the same argument, \(f\) is differentiable with derivative
  \(\overline{g'(\bar{z})}\).
  Therefore, \(f(z)\) and \(\overline{f(\bar{z})}\) are simultaneously
  analytic.
  \par\smallskip
  We could also use the Cauchy-Riemann equations.
  Let \(f(z) = u(x,y) + iv(x,y)\) where \(z = x + iy\) so \(\bar{z} = x - iy\).
  Then \(\overline{f(\bar{z})} = \alpha(x,y) - i\beta(x,y)\) where
  \(\alpha(x,y) = u(x,-y)\) and \(\beta(x,y) = v(x,-y)\).
  In order for both to be analytic, they both need to satisfy the
  Cauchy-Riemann equations.
  That is, \(u_x = v_y\), \(u_y = -v_x\), \(\alpha_x = \beta_y\) and
  \(\alpha_y = -\beta_x\).
  \begin{align*}
    u_x(x,y) & = v_y(x,y)\\
    u_y(x,y) & = -v_x(x,y)\\
    \alpha_x(x,y) & = u_x(x,-y)\\
    \alpha_y(x,y) & = -u_y(x,-y)\\
    -\beta_x(x,y) & = v_x(x,-y)\\
    \beta_y(x,y) & = v_y(x,-y)
  \end{align*}
  Suppose that \(\overline{f(\bar{z})}\) satisfies the Cauchy-Riemann
  equations.
  Then \(\alpha_x = u_x(x,-y) = v_y(x,-y) = \beta_y\) and
  \(\alpha_y = -u_y(x,-y) = v_x(x,-y) = -\beta_x\).
  Therefore,
  \begin{align*}
    u_x(x,-y) & = v_y(x,-y)\\
    u_y(x,-y) & = -v_x(x,-y)
  \end{align*}
  which means \(f(\bar{z})\) satisfies the Cauchy-Riemann equations.
  Now, recall that \(\lvert z\rvert = \lvert\bar{z}\rvert\).
  Since \(f(\bar{z})\) satisfies the Cauchy-Riemann equations, for an
  \(\epsilon > 0\) there exists a \(\delta > 0\) such that when 
  \(0 < \lvert\Delta z\rvert < \delta\),
  \(\lvert f(\bar{z}) - \bar{z}_0\rvert = \lvert f(z) - z_0\rvert < \epsilon\).
  Thus, \(\lim_{\Delta z\to 0}f(z) = z_0\) so \(f(z)\) is analytic if
  \(\overline{f(\bar{z})}\) is analytic.
\item
  Prove that the functions \(u(z)\) and \(u(\bar{z})\) are simultaneously
  harmonic.
  \par\smallskip
  Since \(u\) is the real part of \(f(z)\), \(u(z) = u(x,y)\) where
  \(z = x + iy\).
  Suppose \(u(z)\) is harmonic.
  Then \(u(z)\) satisfies Laplace equation.
  \[
  \nabla^2u(z) = u_{xx} + u_{yy} = 0
  \]
  Now, \(u(\bar{z}) = u(x,-y)\) where
  \(\frac{\partial^2}{\partial x^2}u(\bar{z}) = u_{xx}\) and
  \(\frac{\partial^2}{\partial y^2}u(\bar{z}) = u_{yy}\) so
  \[
  \nabla^2u(\bar{z}) = u_{xx} + u_{yy} = 0.
  \]
  Since \(u(z)\) is harmonic, \(u_{xx} + u_{yy} = 0\) so it follows that
  \(u(\bar{z})\) is harmonic as well.
\item
  Show that a harmonic function satisfies the formal differential equation
  \[
  \frac{\partial^2u}{\partial z\partial\bar{z}} = 0.
  \]
  Let \(u\) be a harmonic.
  Then \(\nabla^2u = 0\).
  \begin{subequations}
    \begin{align}
      \frac{\partial}{\partial\bar{z}}
      & = \frac{1}{2}\Bigl(\frac{\partial}{\partial x} +
        i\frac{\partial}{\partial y}\Bigr)\label{2.1.2.5barz}\\
      \frac{\partial}{\partial z}
      & = \frac{1}{2}\Bigl(\frac{\partial}{\partial x} -
        i\frac{\partial}{\partial y}\Bigr)\label{2.1.2.5z}
    \end{align}
  \end{subequations}
  From \cref{2.1.2.5barz}, we have
  \[
  \frac{1}{2}\Bigl(\frac{\partial}{\partial x} +
  i\frac{\partial}{\partial y}\Bigr)u = \frac{1}{2}(u_x + iu_y).
  \]
  Then we have
  \[
  \frac{\partial^2u}{\partial z\partial\bar{z}} =
  \frac{1}{4}\Bigl(\frac{\partial}{\partial x} -
  i\frac{\partial}{\partial y}\Bigr)(u_x + iu_y) =
  \frac{1}{4}\bigl[u_{xx} + u_{yy} + i(u_{yx} - u_{xy})\bigr]
  \]
  Since \(u\) is a solution to the Laplace equation, \(u\) has continuous first
  and second derivatives.
  That is, \(u\in C^2\) at a minimum.
  By Schwarz's theorem, \(u_{xy} = u_{yx}\) so
  \[
  \frac{\partial^2u}{\partial z\partial\bar{z}} = 0.
  \]
  Schwarz's theorem states that if \(f\) is a function of two variables such
  that \(f_{xy}\) and \(f_{yx}\) both exist and are continuous at some point
  \((x_0,y_0)\), then \(f_{xy} = f_{yx}\).
\end{enumerate}

\subsection{Polynomials}

\subsection{Rational Functions}

\begin{enumerate}
\item
  Use the method of the text to develop
  \[
  \frac{z^4}{z^3 - 1}\qquad\text{and}\qquad\frac{1}{z(z + 1)^2(z + 2)^3}
  \]
  in partial fractions.
  \par\smallskip
  Let \(R(z) = \frac{z^4}{z^3 - 1} = z + \frac{z}{z^3 - 1}\).
  The poles of \(R(z)\) occur when \(z^3 = 1\).
  Then the distinct poles are \(z = 1,e^{2i\pi/3},e^{4i\pi/3}\).
  Let \(H(z) = \frac{z}{z^3 - 1}\), \(z\mapsto\beta_i + 1/w\), and
  \(\beta_i\in\{1,e^{2i\pi/3},e^{4i\pi/3}\}\).
  \begin{align*}
    H(1 + 1/w) & = \frac{w}{3} - \frac{w}{3(3w^2 + 3w + 1)}\\
    H(e^{2i\pi/3} + 1/w)
               & = \frac{w}{3e^{2i\pi/3}} - \frac{w}{3e^{2i\pi/3}
                 (3e^{2i\pi/3}w^2 + 3e^{4i\pi/3}w + 1)}\\
    H(e^{4i\pi/3} + 1/w)
               & = \frac{w}{3e^{4i\pi/3}} - \frac{w}{3e^{4i\pi/3}
                 (3e^{4i\pi/3}w^2 + 3e^{2i\pi/3}w + 1)}\\
    H(\beta_i + 1/w) & = \frac{w}{3\beta_i} - Q(w)
  \end{align*}
  Then \(G_i(w) = \frac{w}{3\beta_i}\) where \(w\mapsto 1/(z - \beta_i)\).
  Now, \(R(z) = G(z) + G_i[1/(z - \beta_i)]\) so we finally have that
  \[
  R(z) = \frac{z^4}{z^3 - 1} = z + \frac{z}{z^3 - 1} =
  z + \sum_{i = 1}^3\frac{1}{3\beta_i(z - \beta_i)}.
  \]
  The second problem's numerator already is of a degree less than the
  denominator so we can proceed at once.
  Let \(R(z) = \frac{1}{z(z + 1)^2(z + 2)^3}\).
  The poles of \(R(z)\) are \(\beta_i\in\{0,-1,-2\}\) and
  \(z\mapsto \beta_i + 1/w\).
  \begin{align*}
    R(1/w) & = \frac{w}{(1/w + 1)^2(1/w + 2)^3}\\
           & = \frac{w^6}{(w + 1)^2(2w + 1)^3}\\
           & = \frac{w}{8} + Q(w)\\
    R(1/w - 1) & = \frac{w^6}{(1 - w)(w + 1)^3}\\
           & = 2w - w^2 + Q(w)\\
    R(1/w - 2) & = \frac{w^6}{(1 - 2w)(w - 1)^2}\\
           & = -\frac{17w}{8} - \frac{5w^2}{4} - \frac{w^3}{2} + Q(w)
  \end{align*}
  Therefore, we can write
  \[
  R(z) = \frac{1}{z(z + 1)^2(z + 2)^3} = \frac{1}{8z} + \frac{2}{z + 1} -
  \frac{1}{(z + 1)^2} - \frac{17}{8(z + 2)} - \frac{5}{4(z + 2)^2} -
  \frac{1}{2(z + 2)^3}
  \]
\item
  Use the formula in the preceding exercise to prove that there exists a
  unique polynomial \(P\) or degree \(< n\) with given values \(c_k\) at the
  points \(\alpha_k\) (Lagrange's interpolation polynomial).
\item
  What is the general form of a rational function which has absolute value
  \(1\) on the circle \(\lvert z\rvert = 1\)?
  In particular, how are the zeros and poles related to each other?
\item
  If a rational function is real on \(\lvert z\rvert = 1\), how are the zeros
  and poles situated?
\item
  If \(R(z)\) is a rational function of order \(n\), how large and how small
  can the order of \(R'(z)\) be?
\end{enumerate}

\section{Elementary Theory of Power Series}

\subsection{Sequences}

\subsection{Series}

\subsection{Uniform Convergence}

\begin{enumerate}
\item
  Prove that a convergent sequence is bounded.
  \par\smallskip
  Let \(\{a_n\}\) be a convergent sequence and \(\lim_{n\to\infty}a_n = a\).
  Let \(\epsilon = 1\).
  Then there exists an \(n > N\) such that \(\lvert a_n - a\rvert < 1\).
  \begin{align*}
    \lvert a_n\rvert & = \lvert a_n - a + a\rvert\\
    \intertext{By the triangle inequality, we have}
                     & \leq \lvert a_n - a\rvert + \lvert a\rvert\\
    \lvert a_n\rvert - \lvert a\rvert & \leq \lvert a_n - a\rvert
  \end{align*}
  Therefore, we have that
  \begin{align*}
    \lvert a_n\rvert - \lvert a\rvert & \leq\lvert a_n - a\rvert < 1\\
    \lvert a_n\rvert & < 1 + \lvert a\rvert
  \end{align*}
  For all \(n > N\), \(\lvert a_n\rvert < 1 + \lvert a\rvert\) so let
  \(A = \max\bigl\{1 + \lvert a\rvert,\lvert a_1\rvert,\ldots,
  \lvert a_N\rvert\bigr\}\).
  Thus, \(\lvert a_n\rvert < A\) for some finite \(A\) and hence \(\{a_n\}\) is
  bounded by \(A\).
\item
  If \(\lim_{n\to\infty}z_n = A\), prove that
  \[
  \lim_{n\to\infty}\frac{1}{n}(z_1 + z_2 + \cdots + z_n) = A.
  \]
  Given \(\epsilon > 0\) there exists some \(n > N\) such that
  \[
  \lvert z_n - A\rvert < \frac{N\epsilon}{2}.
  \]
  Now, since \(\lim_{n\to\infty}z_n = A\) converges, it is Cauchy.
  Therefore, there exists \(n,m > N\) such that
  \[
  \lvert z_m - z_n\rvert < \frac{N\epsilon}{2}.
  \]
  Repeating this we have that \(\lvert z_1 + \cdots + z_n - nA\rvert\) or
  \(\lvert 1/n(z_1 + \cdots + z_{n - 1}) - A + (z_n - A)/n\rvert\).
  For a fixed \(N\), we can find \(n\) such that
  \[
  \sum_{i = 1}^{n - 1}\lvert z_i - A\rvert < \frac{N\epsilon}{2}
  \]
  We now have that
  \begin{align*}
    \lvert 1/n(z_1 + \cdots + z_{n - 1}) - A + (z_n - A)/n\rvert
    & \leq \Bigl\lvert 1/n\sum_{i = 1}^{n - 1}(z_i - A)\Bigr\rvert +
      1/n\lvert z_n - A\rvert\eqnumtag\label{2.2.3.2}\\
    & \leq 1/n\sum_{i = 1}^{n - 1}\lvert z_i - A\rvert +
      1/n\lvert z_n - A\rvert\\
    & < 1/n\frac{N\epsilon}{2} + 1/n\frac{N\epsilon}{2}\\
    & < \epsilon
  \end{align*}
  \Cref{2.2.3.2} can be written as
  \(\lvert 1/n(z_1 + \cdots + z_n) - A\rvert < \epsilon\) so
  \[
  \lim_{n\to\infty}1/n(z_1 + \cdots + z_n) = A.
  \]
\item
  Show that the sum of an absolutely convergent series does not change if the
  terms are rearranged.
  \par\smallskip
  Let \(\sum a_n\) be an absolutely convergent series and \(\sum b_n\) be its
  rearrangement.
  Since \(\sum a_n\) converges absolutely, for \(\epsilon > 0\), there exists
  a \(n > N\) such that \(\lvert s_n - A\rvert < \epsilon/2\) where \(s_n\) is
  the \(n\)th partial sum.
  Let \(t_n\) be the \(n\)th partial sum of \(\sum b_n\).
  Then for some \(n > N\)
  \begin{align*}
    \lvert t_n - A\rvert & = \lvert t_n - s_n + s_n - A\rvert\\
                         & \leq \lvert t_n - s_n\rvert + \lvert s_n - A\rvert\\
                         & < \lvert t_n - s_n\rvert + \frac{\epsilon}{2}
  \end{align*}
  Since \(\sum a_n\) is absolutely convergent,
  \(\sum_{k = n + 1}^{\infty}\lvert a_k\rvert\) converges to zero.
  Let the remainder be \(r_n\).
  Then for some \(N > n,n_1\), \(\lvert r_n - 0\rvert < \epsilon/2\).
  Let \(M = \max\{k_1,k_2,\ldots,k_N\}\).
  Then for some \(n > M\), we have
  \[
  \lvert t_n - s_n\rvert = \Bigl\lvert\sum_n^N a_n\Bigr\rvert\leq
  \sum\lvert a_n\rvert\leq\sum_{k = n + 1}^{\infty}\lvert a_n\rvert = r_n <
  \frac{\epsilon}{2}
  \]
  Thus, \(\lvert t_n - s_n\rvert < \epsilon\) and a rearrangement of an
  absolutely convergent series does not changes its sum.
\item
  Discuss completely the convergence and uniform convergence of the sequence
  \(\{nz^n\}_{n = 1}^{\infty}\).
  \par\smallskip
  Consider when \(\lvert z\rvert < 1\).
  Then \(z^n = \frac{1}{w^n}\) where \(\lvert w\rvert > 1\).
  By the ratio test, we have
  \[
  \lim_{n\to\infty}\Bigl\lvert\frac{(n + 1)w^n}{nw^{n + 1}}\Bigr\rvert =
  \frac{1}{\lvert w\rvert}\lim_{n\to\infty}\frac{n + 1}{n} =
  \frac{1}{\lvert w\rvert}
  \]
  In order for convergence, the ratio test has to be less than one.
  \[
  \frac{1}{\lvert w\rvert} = \lvert z\rvert
  \]
  which is less than one by our assumption so \(\{nz^n\}\) converges
  absolutely in the disc less than one.
  Now, let's consider \(\lvert z\rvert\geq 1\).
  By the ratio test, we get
  \(\lim_{n\to\infty}\lvert a_{n + 1}/a_n\rvert = \lvert z\rvert\geq 1\) by our
  assumption.
  When the limit is one, we can draw no conclusion about convergence, but when
  the limit is greater than one, the sequence diverges.
  For \(\lvert z\rvert < 1\), \(\epsilon > 0\), and \(n > N\),
  \(\lvert nz^n - 0\rvert < \epsilon\) for uniform convergence.
  Take \(z = 9/10\), \(n = 100\), and \(\epsilon = 0.001\).
  Then
  \begin{gather*}
    \lvert nz^n\rvert = n\lvert z\rvert^n < \epsilon\\
    \lvert z\rvert^n < \frac{\epsilon}{n}\\
    0.000026 \not < 0.00001
  \end{gather*}
  Thus, the sequence is not uniformly convergent in the disc with radius less
  than one.
  Let's consider the closed disc \(\lvert z\rvert\leq R\) where \(R\in(0,1)\).
  Now \(\lvert nz^n\rvert\) is bounded above by a convergent geometric series,
  say \(\sum r^n\) where \(\lvert r\rvert < 1\).
  Then \(\lvert nz^n\rvert < ar^n\) for \(\lvert z\rvert\leq R\) and \(a\) a
  real constant.
  Let \(M_n = ar^n\) where \(M_n\) is the \(M\) in the Weierstrass M-test.
  Thus, \(\{nz^n\}\) is uniformly convergent in a closed disc less than one.
\item
  Discuss the uniform convergence of the series
  \[
  \sum_{n = 1}^{\infty}\frac{x}{n(1 + nx^2)}
  \]
  for real values of \(x\).
  \par\smallskip
  By the AM-GM inequality, \((x + y)/2\geq \sqrt{xy}\), we have
  \[
  1 + nx^2\geq 2\lvert x\rvert\sqrt{n}
  \]
  or \(\frac{1}{2\lvert x\rvert\sqrt{n}}\geq\frac{1}{1 + nx^2}\).
  Let \(f_n(x) = \frac{x}{n(1 + nx^2)}\).
  Then
  \[
  \lvert f_n(x)\rvert\leq \Bigl\lvert\frac{x}{2xn^{3/2}}\Bigr\rvert =
  \Bigl\lvert\frac{1}{2n^{3/2}}\Bigr\rvert = M_n
  \]
  For a fixed \(x\), \(\sum\lvert f_n(x)\rvert\leq M_n < \infty\) so
  \(\sum\lvert f_n(x)\rvert\) is absolutely convergent.
  Thus, \(\sum f_n(x)\) is pointwise convergent to \(f(x)\).
  Let \(\epsilon > 0\) be given and \(s_n = \sum_{k = 1}^nf_k(x)\) be \(n\)th
  partial sum.
  Let \(n > N\) such that
  \[
  \lvert f(x) - s_n\rvert = \Bigl\lvert\sum_{k = 1}^{\infty}f_k(x) -
  \sum_{k = 1}^nf_k(x)\Bigr\rvert =
  \Bigl\lvert\sum_{k = n + 1}^{\infty}f_k(x)\Bigr\rvert\leq
  \sum_{k = n + 1}^{\infty}\lvert f_k(x)\rvert
  \]
  Since \(\sum M_k\) converges to some limit, for \(n\) sufficiently large,
  \(\sum_{k = n + 1}^{\infty}M_k < \epsilon\).
  Select \(N\) such that this is true.
  Then
  \[
  \lvert f(x) - s_n\rvert\leq\sum_{k = n + 1}^{\infty}\lvert f_k(x)\rvert\leq
  \sum_{k = n + 1}^{\infty}M_k < \epsilon
  \]
  Therefore, \(\sum f_n(x)\) where \(f_n(x) = \frac{x}{n(1 + nx^2)}\) is
  uniformly convergent by the Weierstrass M-test.
\item
  If \(U = u_1 + u_2 + \cdots,\) \(V = v_1 + v_2 + \cdots\) are convergent
  series, prove that
  \(UV = u_1v_1 + (u_1v_2 + u_2v_2) + (u_1v_3 + u_2v_2 + u_3v_1) + \cdots\)
  provided that at least one of the series is absolutely convergent.
  (It is easy if both series are absolutely convergent.
  Try to rearrange the proof so economically that the absolute convergence of
  the second series is not needed.)
\end{enumerate}

\subsection{Power Series}

\begin{enumerate}
\item
  Expand \((1 - z)^{-m}\), \(m\) a positive integer, in powers of \(z\).
  \par\smallskip
  The Binomial theorem states that
  \((1 + x)^n = \sum_{k = 0}^{\infty}\binom{n}{k}x^k\).
  In our case, we have
  \[
  (1 - z)^{-m} = \sum_{k = 0}^{\infty}\binom{-m}{k}(-z)^k\eqnumtag
  \label{2.2.4.1}
  \]
  where \(\binom{-m}{k} = (-1)^k\binom{m + k - 1}{k}\).
  Then \cref{2.2.4.1} can be written as
  \[
  (1 - z)^{-m} = \sum_{k = 0}^{\infty}\binom{m + k - 1}{k}z^k = 1 + mz +
  \frac{m(m + 1)}{2!}z^2 + \cdots.
  \]
\item
  Expand \(\frac{2z + 3}{z + 1}\) in powers of \(z - 1\).
  What is the radius of convergence?
  \par\smallskip
  Let's just consider \(\frac{1}{z + 1}\) for the moment.
  \[
  \frac{1}{z + 1} = \frac{1}{z - 1 + 2} = \frac{1/2}{1 + \frac{z + 1}{2}} =
  \frac{1}{2}\sum_{n = 0}^{\infty}(-1)^n\Bigl(\frac{z + 1}{2}\Bigr)^n
  \]
  From the full expressing, we obtain
  \[
  \frac{2z + 3}{z + 1} = \frac{2z + 3}{2}
  \sum_{n = 0}^{\infty}(-1)^n\Bigl(\frac{z + 1}{2}\Bigr)^n.
  \]
  The radius of convergence can be found by
  \(1/R = \limsup_{n\to\infty}\sqrt[n]{\lvert a_n\rvert}\).
  Therefore, the radius of convergence is
  \[
  R = 1/\limsup_{n\to\infty}\sqrt[n]{\Bigl\lvert (-1)^n
    \frac{1}{2^n}\Bigr\rvert} = \lvert 2\rvert = 2
  \]
\item
  Find the radius of convergence of the following power series:
  \[
  \sum n^pz^n,\quad\sum\frac{z^n}{n!},\quad\sum n!z^n,\quad\sum q^{n^2}z^n,
  \quad\sum z^{n!}
  \]
  where \(\lvert q\rvert < 1\).
  \par\smallskip
  For \(\sum n^pz^n\), we can use the inverse of argument of the ratio test to
  determine the radius of convergence; that is,
  \[
  R = \lim_{n\to\infty}\Big\lvert\frac{n^p}{(n + 1)^p}\Bigr\rvert
  = \lim_{n\to\infty}\frac{n^p}{(n + 1)^p} = 1
  \]
  For \(\sum\frac{z^n}{n!}\), we can use the fact that the sum is \(e^z\) which
  is entire or the method used previously.
  Since \(e^z\) is entire, the radius of convergence is \(R = \infty\).
  \[
  R = \lim_{n\to\infty}\Big\lvert\frac{(n + 1)!}{n!}\Bigr\rvert
  = \lim_{n\to\infty}\frac{n!(n + 1)}{n!} = \infty
  \]
  For \(\sum n!z^n\), we use the modified ratio test again.
  \[
  R = \lim_{n\to\infty}\Bigl\lvert\frac{n!}{(n + 1)!}\Bigr\rvert = 0
  \]
  For \(\sum q^{n^2}z^n\), we will use the ratio test.
  \[
  R = \lim_{n\to\infty}\Bigl\lvert\frac{q^{n^2}}{q^{(n + 1)^2}}\Bigr\vert =
  \frac{1}{\lvert q\rvert}
  \lim_{n\to\infty}\Bigl\lvert\frac{1}{q^{2n}}\Bigr\rvert =
  \infty
  \]
  For \(\sum z^{n!}\), we will use the root test.
  \[
  R = 1/\limsup_{n\to\infty}\sqrt[n]{\lvert z^{(n - 1)!}\rvert^n} =
  1/\limsup_{n\to\infty}\lvert z\rvert^{(n - 1)!}
  \]
  When \(\lvert z\rvert < 1\), \(R = \infty\), and when \(\lvert z\rvert > 1\),
  \(R = 0\).
\item
  If \(\sum a_nz^n\) has a radius of convergence \(R\), what is the radius of
  convergence of \(\sum a_nz^{2n}\)? of \(\sum a_n^2z^n\)?
  \par\smallskip
  Since \(\sum a_nz^n\) has a radius of convergence \(R\),
  \[
  R = \lim_{n\to\infty}\Bigl\lvert\frac{a_n}{a_{n + 1}}\Bigr\rvert.
  \]
  For \(\sum a_nz^{2n} = z^2\sum a_nz^n\), we have
  \[
  \lvert z\rvert^2\lim_{n\to\infty}\Bigl\lvert\frac{a_n}{a_{n + 1}}\Bigr\rvert
  = \lvert z\rvert^2R
  \]
  so the radius of convergence is \(\sqrt{R}\).
  For \(\sum a_n^2z^n\), we have
  \[
  \lim_{n\to\infty}\Bigl\lvert\frac{a_n}{a_{n + 1}}\Bigr\rvert^2 = R^2.
  \]
\item
  If \(f(z) = \sum a_nz^n\), what is \(\sum n^3a_nz^n\)?
  \par\smallskip
  Let's write out the first few terms of
  \[
  \sum n^3a_nz^n = a_1z + 8a_2z^2 + 27a_3z^3 + 64a_4z^4 + \cdots
  \]
  Let's consider the first three derivatives of \(f(z)\).
  \begin{align*}
    f'(z) & = \sum na_nz^{n - 1}\\
    zf'(z) & = \sum na_nz^n\\
          & = a_1z + 2a_2z^2 + 3a_3z^3 + \cdots\eqnumtag\label{2.2.4.51}\\
    f''(z) & = \sum n(n - 1)a_nz^{n - 2}\\
    z^2f''(z) & = \sum n(n - 1)a_nz^n\\
          & = 2a_2z^2 + 6a_3z^3 + 12a_4z^4 + \cdots\eqnumtag\label{2.2.4.52}\\
    f'''(z) & = \sum n(n - 1)(n - 2)a_nz^{n - 3}\\
    z^3f'''(z) & = \sum n(n - 1)(n - 2)a_nz^n\\
          & = 6a_3z^3 + 24a_4z^4 + 60a_5z^5 + \cdots\eqnumtag\label{2.2.4.53}
  \end{align*}
  If we add \cref{2.2.4.51,2.2.4.52,2.2.4.53}, we have
  \[
  zf'(z) + z^2f''(z) + z^3f'''(z) = a_1z + 4a_2z^2 + 15a_3z^3 + \cdots\neq
  \sum n^3a_nz^n
  \]
  However, consider \(3z^2f''(z) = 6a_2z^2 + 18a_3z^3 + 36a_4z^4 + \cdots\).
  Then
  \[
  zf'(z) + 3z^2f''(z) + z^3f'''(z) = a_1z + 8a_2z^2 + 27a_3z^3 + 64a_4z^4
  \cdots = \sum n^3a_nz^n.
  \]
\item
  If \(\sum a_nz^n\) and \(\sum b_nz_n\) have radii of convergence \(R_1\) and
  \(R_2\), show that the radii of convergence of \(\sum a_nb_nz^n\) is at least
  \(R_1R_2\).
  \par\smallskip
  Let \(\epsilon > 0\) be given. Then there exists \(n > N\) such that 
  \[
  \lvert a_n\rvert^{1/n} < 1/R_1 + \epsilon,\qquad \lvert b_n\rvert^{1/n} <
  1/R_2 + \epsilon
  \]
  since \(\limsup_{n\to\infty}\lvert a_n\rvert^{1/n} = 1/R_1\) so
  \(\lvert a_n\rvert^{1/n} < 1/R_1 + \epsilon\) and similarly for \(b_n\).
  Multiplying we obtain
  \[
  \lvert a_nb_n\rvert^{1/n} < \frac{1}{R_1R_2} + \epsilon(1/R_1 + 1/R_2) +
  \epsilon^2
  \] 
  Then 
  \[
  \frac{1}{R}\leq \frac{1}{R_1R_2}\Rightarrow R_1R_2\leq R
  \]
\item
  If \(\lim_{n\to\infty}\lvert a_n\rvert/\lvert a_{n + 1}\rvert = R\), prove
  that \(\sum a_nz^n\) has a radius of convergence of \(R\).
  \par\smallskip
  Let \(\epsilon > 0\) be given.
  Suppose \(\lvert z\rvert < R\).
  Pick \(\epsilon\) such that \(\lvert z\rvert < R - \epsilon\).
  Then for some \(n > N\)
  \begin{alignat*}{2}
    \biggl\lvert\Bigl\lvert\frac{a_n}{a_{n + 1}}\Bigr\rvert - R\biggr\rvert
    & \leq R - \Bigl\lvert\frac{a_n}{a_{n + 1}}\Bigr\rvert &&{} < \epsilon\\
    R - \epsilon & < \Bigl\lvert\frac{a_n}{a_{n + 1}}\Bigr\rvert\eqnumtag
    \label{2.2.4.7}
  \end{alignat*}
  For \(n > N\), we can write
  \[
  \Bigl\lvert\frac{a_N}{a_n}\Bigr\rvert =
  \Bigl\lvert\frac{a_Na_{N + 1}\cdots a_{n - 1}}{a_{N + 1}a_{N + 2}\cdots a_n}
  = \Bigl\lvert\frac{a_N}{a_{N + 1}}\frac{a_{N + 1}}{a_{N + 2}}\cdots \frac{a_{n - 1}}{a_n}\Bigr\rvert\eqnumtag\label{2.2.4.72}
  \]
  For \(n > N\), we have that from \cref{2.2.4.7},
  \(R - \epsilon < \frac{a_N}{a_{N + 1}}\).
  Thus, we can \cref{2.2.4.72} as
  \begin{align*}
    (R - \epsilon)^{n - N} & < \Bigl\lvert\frac{a_N}{a_n}\Bigr\rvert\\
    \lvert a_n\rvert & < \frac{\lvert a_N\rvert}{(R - \epsilon)^{n - N}}\\
    \lvert a_nz^n\rvert & < \lvert a_Nz^N\rvert\Bigl(\frac{\lvert z\rvert}
                          {R - \epsilon}\Bigr)^{n - N}
  \end{align*}
  Since \(\epsilon\) was chosen such that \(\lvert z\rvert < R - \epsilon\), we
  that \(\frac{\lvert z\rvert}{R - \epsilon} < 1\) and
  \[
  \lvert a_nz^n\rvert < \lvert a_Nz^N\rvert
  \]
  where \(\lvert a_Nz^N\rvert < \infty\) since it is a convergent geometric
  series.
  Therefore, \(\sum a_nz^n\) converges absolutely with a radius of convergence
  of \(R\).
\item
  For what values of \(z\) is
  \[
  \sum_{n = 0}^{\infty}\Bigl(\frac{z}{1 + z}\Bigr)^n
  \]
  convergent?
  \par\smallskip
  In order for series to converge
  \(\limsup_{n\to\infty}\sqrt[n]{\lvert a_n\rvert} < 1\).
  Then
  \[
  \limsup_{n\to\infty}\sqrt[n]{\lvert z/(z+1)\rvert^n} =
  \Bigl\lvert\frac{z}{z + 1}\Bigr\rvert < 1
  \]
  or \(\lvert z\rvert^2 < (1 + z)(1 + \bar{z}) = 1 + 2\Re\{z\} +
  \lvert z\rvert^2\) so the series converges when
  \[
  0 < 1 + 2\Re\{z\}.
  \]
\item
  Same question for
  \[
  \sum_{n = 0}^{\infty}\frac{z^n}{1 + z^{2n}}.
  \]
  Consider the following two equations:
  \begin{align*}
    \lvert 1\rvert & = \lvert 1 + z^{2n} - z^{2n}\rvert\\
                   & \leq \lvert 1 + z^{2n}\rvert + \lvert z^{2n}\rvert\\
    1 - \lvert z^{2n}\rvert & \leq \lvert 1 + z^{2n}\rvert\eqnumtag
                              \label{2.2.4.9a}\\
    \lvert z^{2n}\rvert & = \lvert 1 - 1 + z^{2n}\rvert\\
                   & \leq \lvert 1 + z^{2n}\rvert + 1\\
    \lvert z^{2n}\rvert - 1 & \leq \lvert 1 + z^{2n}\rvert\eqnumtag
                              \label{2.2.4.9b}\\
  \end{align*}
  From \cref{2.2.4.9a,2.2.4.9b}, the triangle inequality, we have that
  \[
  \bigl\lvert 1 - \lvert z^{2n}\rvert\bigr\rvert\leq\lvert 1 + z^{2n}\rvert.
  \]
  There exists an \(m > 1\) such that
  \[
  \frac{\lvert z^{2n}\rvert}{m}\leq
  \bigl\lvert 1 - \lvert z^{2n}\rvert\bigr\rvert.
  \]
  By the root test,
  \[
  \limsup_{n\to\infty}\sqrt[n]{\frac{m\lvert z\rvert^n}{\lvert z^2\rvert^n}} =
  \limsup_{n\to\infty}\frac{\sqrt[n]{m}}{\lvert z\rvert} =
  \frac{1}{\lvert z\rvert} < 1
  \]
  When \(\lvert z\rvert > 1\), the convergence of the ratio test
  \(\frac{1}{\lvert z\rvert} < 1\) leads to \(\lvert z\rvert > 1\).
  If \(\lvert z\rvert < 1\), then \(1/\lvert z\rvert > 1\) where we can write
  \(1/\lvert z\rvert = \lvert z_1\rvert\).
  Since the choice dummy variables is arbitrary, \(\lvert z\rvert < 1\).
  In other words, the series will converge when \(\lvert z\rvert > 1\) or
  \(\lvert z\rvert < 1\).
  Suppose \(\lvert z\rvert = 1\).
  Then by the limit test,
  \[
  \lim_{n\to\infty}\frac{1}{1^n + 1^{-n}} = \frac{1}{2}\neq 0;
  \]
  therefore, the series diverges.
\end{enumerate}

\subsection{Abel's Limit Theorem}

\section{The Exponential and Trigonometric Functions}

\subsection{The Exponential}

\subsection{The Trigonometric Functions}

\begin{enumerate}
\item
  Find the values of \(\sin(i)\), \(\cos(i)\), and \(\tan(1 + i)\).
  \par\smallskip
  For \(\sin(i)\), we can use the identity
  \(\sin(z) = \frac{e^{iz} - e^{-iz}}{2i}\).
  Then
  \[
  \sin(i) = \frac{e^{-1} - e^{1}}{2i} = i\frac{e^{1} - e^{-1}}{2} = i\sinh(1).
  \]
  Similarly, for \(\cos(i)\), we have
  \[
  \cos(i) = \frac{e^{-1} + e^{1}}{2} = \frac{e^{1} + e^{-1}}{2} = \cosh(1).
  \]
  For \(\tan(1 + i)\), we can use the identity
  \(\tan(z) = -i\frac{e^{iz} - e^{-iz}}{e^{iz} + e^{-iz}}\).
  Then
  \[
  \tan(1 + i) = -i\frac{e^{i - 1} - e^{1 - i}}{e^{i - 1} + e^{1 - i}} =
  -i\tanh(i - 1).
  \]
\item
  The hyperbolic cosine and sine are defined as
  \(\cosh(z) = \frac{e^z + e^{-z}}{2}\) and
  \(\sinh(z) = \frac{e^z - e^{-z}}{2}\).
  Express them through \(\cos(iz)\) and \(\sin(iz)\).
  Derive the addition formulas, and formulas for \(\cosh(2z)\) and
  \(\sinh(2z)\).
  \par\smallskip
  For the first part, we have
  \begin{align*}
    \cos(iz) & = \frac{e^{-z} + e^z}{2}\\
             & = \cosh(z)\\
    \sin(iz) & = \frac{e^{-z} - e^z}{2i}\\
             & = i\frac{e^z - e^{-z}}{2}\\
             & = i\sinh(z)
  \end{align*}
  For \(\cosh\), we have that the addition formula is
  \begin{align*}
    \cosh(a + b) & = \cos[i(a + b)]\\
                 & = \frac{e^{a + b} + e^{-(a + b)}}{2}\\
                 & = \frac{2e^{a + b} + 2e^{-(a + b)}}{4}\\
                 & = \bigl(e^{a + b} + e^{a - b} + e^{b - a} + e ^{-(a + b)} +
                   e^{a + b} - e^{a - b} - e^{b - a} + e^{-(a + b)}\bigr)/4\\
                 & = \frac{e^a + e^{-a}}{2}\frac{e^b + e^{-b}}{2} +
                   \frac{e^a - e^{-a}}{2}\frac{e^b - e^{-b}}{2}\\
                 & = \cosh(a)\cosh(b) + \sinh(a)\sinh(b)
  \end{align*}
  For \(\sinh\), we have that the addition formula is
  \begin{align*}
    \sinh(a + b) & = -i\sin[i(a + b)]\\
                 & = \frac{e^{-(a + b)} - e^{a + b}}{-2}\\
                 & = \frac{2e^{a + b} - 2e^{-(a + b)}}{4}\\
                 & = \bigl(e^{a + b} + e^{a - b} - e^{b - a} - e ^{-(a + b)} +
                   e^{a + b} - e^{a - b} + e^{b - a} - e^{-(a + b)}\bigr)/4\\
                 & = \frac{e^a - e^{-a}}{2}\frac{e^b + e^{-b}}{2} +
                   \frac{e^a + e^{-a}}{2}\frac{e^b - e^{-b}}{2}\\
                 & = \sinh(a)\cosh(b) + \cosh(a)\sinh(b)
  \end{align*}
  For the double angle formulas, recall that
  \(\cos(2z) = \cos^2(z) - \sin^2(z) = 2\cos^2(z) - 1 = 1 - 2\sin^2(z)\) and
  \(\sin(2z) = 2\sin(z)\cos(z)\).
  Therefore, we have
  \begin{align*}
    \cosh(2z) & = \cos(2iz)\\
              & = \cos^2(iz) - \sin^2(iz)\\
              & = \Bigl(\frac{e^z + e^{-z}}{2}\Bigr)^2 +
                \Bigl(\frac{e^z - e^{-z}}{2}\Bigr)^2\\
              & = \cosh^2(z) + \sinh^2(z)\\
    \cosh(2z) & = 2\cos^2(iz) - 1\\
              & = 2\cosh^2(z) - 1\\
    \cosh(2z) & = 1 - 2\sin^2(iz)\\
              & = 1 - 2\sinh^2(z)\\
    \sinh(2z) & = -i\sin(2iz)\\
              & = -2i\sin(iz)\cos(iz)\\
              & = 2\frac{e^z - e^{-z}}{2}\frac{e^z + e^{-z}}{2}\\
              & = 2\sinh(z)\cosh(z)
  \end{align*}
\item
  Use the addition formulas to separate \(\cos(x + iy)\) and \(\sin(x + iy)\)
  in real and imaginary parts.
  \begin{align*}
    \cos(x + iy) & = \cos(x)\cos(iy) - \sin(x)\sin(iy)\\
                 & = \cos(x)\cosh(y) - i\sin(x)\sinh(y)\\
    \sin(x + iy) & = \sin(x)\cos(iy) + \sin(iy)\cos(x)\\
                 & = \sin(x)\cosh(y) + i\sinh(y)\cos(x)
  \end{align*}
\item
  Show that
  \[
  \lvert\cos(z)\rvert^2 = \sinh^2(y) + \cos^2(x) = \cosh^2(y) - \sin^2(x) =
  (\cosh(2y) + \cos(2x))/2
  \]
  and
  \[
  \lvert\sin(z)\rvert^2 = \sinh^2(y) + \sin^2(x) = \cosh^2(y) - \cos^2(x) =
  (\cosh(2y) - \cos(2x))/2.
  \]
  For the identities, recall that \(\cosh^2(z) - \sinh^2(z) = 1\) and
  \(\cos^2(z) + \sin^2(z) = 1\).
  Then for the first identity, we have
  \begin{align*}
    \lvert\cos(z)\rvert^2 & = \cos(z)\cos(\bar{z})\\
                          & = \bigl[\cos(x)\cosh(y) - i\sin(x)\sinh(y)\bigr]
                            \bigl[\cos(x)\cosh(y) + i\sin(x)\sinh(y)\bigr]\\
                          & = \cos^2(x)\cosh^2(y) + \sin^2(x)\sinh^2(y)\\
                          & = \cos^2(x)(1 + \sinh^2(y)) + \sin^2(x)\sinh^2(y)\\
                          & = \cos^2(x) + \sinh^2(y)\\
    \lvert\cos(z)\rvert^2 & = \cos^2(x)\cosh^2(y) + \sin^2(x)(\cosh^2(y) - 1)\\
                          & = \cosh^2(y) - \sin^2(x)\\
    \lvert\cos(z)\rvert^2 & = \\
                          & = (\cosh(2y) + \cos(2x))/2\\
    \lvert\sin(z)\rvert^2 & = \sin(z)\sin(\bar{z})\\
                          & = \bigl[\sin(x)\cosh(y) + i\sinh(y)\cos(x)\bigr]
                            \bigl[\sin(x)\cosh(y) - i\sinh(y)\cos(x)\bigr]\\
                          & = \sin^2(x)\cosh^2(y) + \sinh^2(y)\cos^2(x)\\
                          & = \sin^2(x)(1 + \sinh^2(y)) + \sinh^2(y)\cos^2(x)\\
                          & = \sin^2(x) + \sinh^2(y)\\
    \lvert\sin(z)\rvert^2 & = \sin^2(x)\cosh^2(y) + (\cosh^2(y) - 1)\cos^2(x)\\
                          & = \cosh^2(y) - \cos^2(x)\\
    \lvert\sin(z)\rvert^2 & = \\
                          & = (\cosh(2y) - \cos(2x))/2
  \end{align*}
\end{enumerate}

\subsection{Periodicity}

\subsection{The Logarithm}

\begin{enumerate}
\item
  For real \(y\), show that every remainder in the series for \(\cos(y)\) and
  \(\sin(y)\) has the same sign as the leading term (this generalizes the
  inequalities used in the periodicity proof).
  \par\smallskip
  The series for both cosine and sine are
  \begin{align*}
    \cos(y) & = \sum_{k = 0}^{\infty}(-1)^k\frac{y^{2k}}{(2k)!}\\
            & = 1 - \frac{y^2}{2!} + \frac{y^4}{4!} - \cdots\\
    \sin(y) & = \sum_{k = 0}^{\infty}(-1)^k\frac{y^{2k + 1}}{(2k + 1)!}\\
            & = y - \frac{y^3}{3!} + \frac{y^6}{6!} - \cdots
  \end{align*}
  We can write Taylor's formula as \(f(y) = T_n(y) + R_n(y)\) where
  \[
  f(y) = \sum_{k = 0}^n\frac{f^{(k)}(0)}{k!}y^k + \frac{1}{k!}\int_0^y(y - t)^k
  f^{(k + 1)}(t)dt.
  \]
  Now, we can write cosine and sine of \(y\) as
  \begin{align*}
    \cos(y) & = \sum_{k = 0}^n\frac{(-1)^ky^{2k}}{(2k)!} + \frac{1}{n!}
              \int_0^y(y - t)^n\cos^{n + 1}(t)dt\\
    \sin(y) & = \sum_{k = 0}^{n - 1}\frac{(-1)^ky^{2k + 1}}{(2k + 1)!} +
              \frac{1}{n!}\int_0^y(y - t)^n\sin^n(t)dt\\
    \intertext{For cosine and sine, let \(n = 2m\) and \(n = 2m - 1\),
    respectively.
    Then}
    \cos(y) & = \sum_{k = 0}^m\frac{(-1)^ky^{2k}}{(2k)!} + \frac{1}{(2m)!}
              \int_0^y(y - t)^{2m}\cos^{2m + 1}(t)dt\\
    \sin(y) & = \sum_{k = 0}^{m - 1}\frac{(-1)^ky^{2k + 1}}{(2k + 1)!} +
              \frac{1}{(2m - 1)!}\int_0^y(y - t)^{2m - 1}\sin^{2m - 1}(t)dt
  \end{align*}
\item
  Prove, for instance, that \(3 < \pi < 2\sqrt{3}\).
\item
  Find the value of \(e^z\) for \(z = -i\pi/2,3i\pi/4,2i\pi/3\).
  \begin{align*}
    e^{-i\pi/2} & = -i\\
    e^{3i\pi/4} & = (-\sqrt{2} + i\sqrt{2})/2\\
    e^{2i\pi/3} & = (-1 + i\sqrt{3})/2
  \end{align*}
\item
  For that values of \(z\) is \(e^z\) equal to \(2,-1,i,-i/2,-1 - i,1 + 2i\)?
  \par\smallskip
  For all problems, \(k\in\mathbb{Z}\).
  \begin{alignat*}{4}
    e^z & = 2 & \qquad & e^z && ={} -1\\
    z & = \log(2) + 2ki\pi & & z && ={} \log(-1)\\
    e^z & = i & & && ={}\log\lvert i\rvert + i(\arg(-1) + 2k\pi)\\
    z & = \log\lvert i\rvert + i(\arg(i) + 2k\pi) & & && = {} i\pi(1 + 2k\pi)\\
    & = \frac{i\pi}{2}(1 + 4k) & & e^z && = \frac{-i}{2}\\
    e^z & = -1 - i & & && = -\log(2) - \frac{i\pi}{2}(1 + 4k)\\
    z & = \log\lvert -1 - i\rvert + i(\arg(-1 - i) + 2k\pi) & &
    e^z && ={} 1 + 2i\\
    & = \log(\sqrt{2}) - \frac{3i\pi}{4} + 2ki\pi & & z
    && ={} \log(\sqrt{5}) + i(\arctan(2) + 2k\pi)\\
    & = \frac{\log(2)}{2} - \frac{3i\pi}{4} + 2ki\pi & &
    && ={} \frac{\log(5)}{2} + i(\arctan(2) + 2k\pi)
  \end{alignat*}
\item
  Find the real and imaginary parts of \(\exp(e^z)\).
  \par\smallskip
  Let \(z = x + iy\).
  Then
  \begin{align*}
    \exp(e^z) & = \exp\bigl[e^x(\cos(y) + i\sin(y)\bigr]\\
              & = \exp(e^x\cos(y))\exp(ie^x\sin(y))\\
              & = \exp(e^x\cos(y))
                \bigl[\cos(e^x\sin(y)) + i\sin(e^x\sin(y))\bigr]\\
    u(x, y) & = \exp(e^x\cos(y))\cos(e^x\sin(y))\\
    v(x, y) & = \exp(e^x\cos(y))\sin(e^x\sin(y))
  \end{align*}
  where \(u(x,y)\) is the real and \(v(x,y)\) is the imaginary part of
  \(\exp(e^z)\).
\item
  Determine all values of \(2^i,i^i,(-1)^{2i}\).
  \par\smallskip
  For all problems, \(k\in\mathbb{Z}\).
  \begin{align*}
    z & = 2^i\\
      & = \exp\bigl[i\log(2)\bigr]\\
      & = \cos(\log(2)) + i\sin(\log(2))\\
    z & = i^i\\
      & = \exp\bigl[i\log(i)\bigr]\\
      & = \exp[-\pi(1 + 4k)/2]\\
    z & = (-1)^{2i}\\
      & = i^{4i}\\
      & = (i^i)^4\\
      & = \exp[-2\pi(1 + 4k)]
  \end{align*}
\item
  Determine the real and imaginary parts of \(z^z\).
  \par\smallskip
  Let \(z = x + iy\) and \(k\in\mathbb{Z}\).
  \begin{align*}
    z^z & = (x + iy)^{x + iy}\\
        & = \exp\bigl[(x + iy)\log(x + iy)\bigr]\\
        & = e^{x/2\log(x^2 + y^2) - y(\arctan(y/x) + 2k\pi)}
          \begin{aligned}
            &
            \Bigl[\cos\bigl(x(\arctan(y/x) + 2k\pi) + y/2\log(x^2 + y^2)\bigr)
            +\\
            &
            i\sin\bigl(x(\arctan(y/x) + 2k\pi) + y/2\log(x^2 + y^2)\bigr)\Bigr]
          \end{aligned}
  \end{align*}
  Thus, the real part is
  \[
  u(x, y) = e^{x/2\log(x^2 + y^2) - y(\arctan(y/x) + 2k\pi)}
  \cos\bigl(x(\arctan(y/x) + 2k\pi) + y/2\log(x^2 + y^2)\bigr)
  \]
  and the imaginary part is
  \[
  v(x, y) = e^{x/2\log(x^2 + y^2) - y(\arctan(y/x) + 2k\pi)}
  \sin\bigl(x(\arctan(y/x) + 2k\pi) + y/2\log(x^2 + y^2)\bigr)
  \]
\item
  Express \(\arctan(w)\) in terms of the logarithm.
  \par\smallskip
  Let \(\arctan(w) = z\).
  Then \(w = \tan(z)\).
  Recall that \(\tan(z) = -i\frac{e^{iz} - e^{-iz}}{e^{iz} + e^{-iz}}\).
  Now, let \(e^{2iz} = x\).
  Then we have the following
  \[
  w = -i\frac{x^2 - 1}{x^2 + 1}
  \]
  which leads to
  \[
  e^{2iz} = \frac{i - w}{i + w}.
  \]
  By taking the \(\log\), we can recover \(z\).
  \begin{align*}
    2iz & = \log(i - w) - \log(i + w)\\
        & = \log(i) + \log(1 + iw) - \log(i) - \log(1 - iw)\\
    z & = \frac{i}{2}\bigl[\log(1 - iw) - \log(1 + iw)\bigr]\\
    \arctan(w) & = z\\
        & = \frac{i}{2}\bigl[\log(1 - iw) - \log(1 + iw)\bigr]
  \end{align*}
\item
  Show how to define the "angles" in a triangle, bearing in mind that they
  should lie between \(0\) and \(\pi\).
  With this definition, prove that the sum of the angles is \(\pi\).
\item
  Show that the roots of the binomial equation \(z^n = a\) are the vertices of
  a regular polygon (equal sides and angles).
\end{enumerate}

%%% Local Variables:
%%% mode: latex
%%% TeX-master: t
%%% End:
