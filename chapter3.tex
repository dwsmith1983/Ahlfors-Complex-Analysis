\chapter{Analytic Functions as Mappings}

\section{Elementary Point Set Topology}

\subsection{Sets and Elements}

\subsection{Metric Spaces}

\begin{exercise}[ref = \arabic{exercisei}]
\item
  \label{3.1.2.1}
  If \(S\) is a metric space with distance function \(d(x,y)\), show that \(S\)
  with the distance function \(\delta(x,y) = d(x,y)/[1 + d(x,y)]\) is also a
  metric space.
  The latter space is bounded in the sense that all distances lie under a fixed
  bound.
  \par\smallskip
  Since \(d(x,y)\) is a metric on \(S\), \(d(x,y)\) satisfies
  \begin{subequations}
    \begin{align}
      d(x, y) & \geq 0,\text{  and zero only when \(x = y\)}\label{3.1.2.1a}\\
      d(x, y) & = d(y, x)\label{3.1.2.1b}\\
      d(x, z) & \leq d(x, y) + d(y, z)\label{3.1.2.1c}
    \end{align}
  \end{subequations}
  By \cref{3.1.2.1a}, for \(x = y\), \(d(x,y) = 0\) so
  \(\delta(x,y) = 0/1 = 0\), and when \(x\neq y\), \(d(x,y) > 0\) so
  \(\delta(x,y) > 0\) since a positive number divided by a positive number is
  positive.
  We have that \(\delta(x,y)\geq 0\) and equal zero if and only if \(x = y\).
  By \cref{3.1.2.1b}, we have
  \begin{align*}
    \delta(x, y) & = \frac{d(x, y)}{1 + d(x, y)}\\
                 & = \frac{d(y, x)}{1 + d(y, x)}\\
                 & = \delta(y, x)
  \end{align*}
  For the triangle inequality, we have
  \begin{align*}
    \frac{d(x, z)}{1 + d(x, z)}
    & \leq \frac{d(x, y)}{1 + d(x, y)} + \frac{d(y, z)}{1 + d(y, z)}\\
    \intertext{Let's multiple through by the product of all three
    denominators.
    After simplifying, we obtain}
    d(x, z) & \leq d(x, y) + d(y,z) + 2d(x, y)d(y, z) + d(x, y)d(y, z)d(x, z)
  \end{align*}
  We have already shown that \(d(x,y)\geq 0\) and zero if and only if
  \(x = y\).
  If \(x = y = z\), the triangle inequality is vacuously true.
  When \(x\neq y\neq z\), the triangle inequality follows since each distance
  is positive and \cref{3.1.2.1c}; that is,
  \[
  \delta(x, z)\leq \delta(x, y) + \delta(y, z).
  \]
\item
  Suppose that there are given two distance functions \(d(x,y)\) and
  \(d_1(x,y)\) on the same space \(S\).
  They are said to be equivalent if they determine the same open sets.
  Show that \(d\) and \(d_1\) are equivalent if to every \(\epsilon > 0\) there
  exists a \(\delta > 0\) such that \(d(x,y) < \delta\) implies
  \(d_1(x,y) < \epsilon\), and vice versa.
  Verify that this condition is fulfilled in \cref{3.1.2.1}.
  \par\smallskip
  Let \(\epsilon,\delta > 0\) be given.
  We can write
  \(\delta(x,y) = \frac{d(x,y)}{1 + d(x,y)} = 1 - \frac{1}{1 + d(x,y)}\).
  We need to find a \(\delta\) such that whenever \(d(x,y) < \delta\),
  \(\delta(x,y) < \epsilon\).
  \begin{align*}
    1 - \frac{1}{1 + d(x, y)} & < \epsilon\\
    d(x, y) & < \frac{\epsilon}{1 - \epsilon}
  \end{align*}
  Let \(\delta = \frac{\epsilon}{1 - \epsilon}\).
  For \(\epsilon < 1\), if \(d(x,y) < \delta = \frac{\epsilon}{1 - \epsilon}\),
  then
  \[
  \delta(x,y) = 1 - \frac{1}{1 + d(x, y)} < 1 -
  \frac{1}{1 + \frac{\epsilon}{1 - \epsilon}} = \epsilon.
  \]
  If \(\epsilon > 1\),
  \(\delta(x,y) = \frac{d(x,y)}{1 + d(x,y)} < 1 < \epsilon\).
  For the reverse implication, we need to find a \(\delta\) such that
  \(d(x,y) < \epsilon\).
  Let \(\delta = \frac{\epsilon}{1 + \epsilon}\).
  For any \(\epsilon > 0\), if
  \(\delta(x,y) < \delta = \frac{\epsilon}{1 + \epsilon}\), then
  \begin{align*}
    \frac{d(x, y)}{1 + d(x, y)} & < \frac{\epsilon}{1 + \epsilon}\\
    d(x, y)(1 + \epsilon) & < \epsilon + \epsilon d(x, y)\\
    d(x, y) & < \epsilon
  \end{align*}
  as was needed to be shown.
  Therefore, \(d(x,y)\) and \(\delta(x,y)\) are equivalent metrics on \(S\).
\item
  Show by strict application of the definition that the closure of
  \(\lvert z - z_0\rvert < \delta\) is \(\lvert z - z_0\rvert\leq\delta\).
\item
  If \(X\) is the set of complex numbers whose real and imaginary parts are
  rational, what is \(\operatorname{Int} X, \bar{X}, \partial X\)?
\item
  It is sometimes typographically simpler to write \(X'\) for \(\sim X\).
  With this notation, how is \(X^{'-'}\) related to \(X\)?
  Show that \(X^{-'-'-'-'} = X^{-'-'}\).
\item
  A set is said to be discrete if all its points are isolated.
  Show that a discrete set in \(\mathbb{R}\) or \(\mathbb{C}\) is countable.
  \par\smallskip
  Let \(S\) be a discrete set in \(\mathbb{R}\) or \(\mathbb{C}\).
  If \(z\in S\), then for some \(\epsilon_i > 0\), for \(i\in\mathbb{Z}\),
  \(N_{\epsilon_i}(z_i)\) is the \(i\)-th neighborhood of \(z_i\).
  Since \(S\) is discrete, there exists an \(\epsilon\) for each \(i\) such
  that the only point in \(N_{\epsilon_i}(z_i)\) is \(z_i\).
  Let \(\epsilon_i\) be this \(\epsilon\). 
  Consider the following function
  \[
  f(i) =
  \begin{cases}
    0, & i = 0\\
    1, & i = 1\\
    2i, & i > 0\\
    2(-i) + 1, & i < 0
  \end{cases}
  \]
  We have put \(i\) in a one-to-one correspondence with \(\mathbb{Z}^+\).
  Therefore, \(S\) is countable.
\item
  Show that the accumulation points of any set form a closed set.
  \par\smallskip
  Let \(E\) be a set.
  Then \(E'\) is the set of accumulation (limit) points.
  If \(z_i\) is a limit point, \(z_i\in E'\).
  Now, \(z_i\) are limit points of \(E\) as well.
  Then \(\{z_i\}\to z\) where \(z\in\bar{E}\).
  Therefore, \(z\in E'\) so \(E'\) is closed.
\end{exercise}

\subsection{Connectedness}

\begin{exercise}[ref = \arabic{exercisei}]
\item
  If \(X\subset S\), show that the relatively open (closed) subsets of \(X\)
  are precisely those sets that can be expressed as the intersection of \(X\)
  with an open (closed) subsets of \(S\).
  \par\smallskip
  Let \(\{U_{\alpha}\}\) be the open sets of \(S\) such that
  \(\bigcup_{\alpha}U_{\alpha} = S\).
  Then
  \(X = X\cap\bigcup_{\alpha}U_{\alpha} = \bigcup_{\alpha}(X\cap U_{\alpha})\).
  Let \(\{A_n\} = \{X\cap U_{\alpha}\}\).
  Then \(A_n\) is relatively open in \(X\) since \(A_n\) belongs to the
  topology of \(X\); that is, for each \(n\), \(A_n\subset X\) and
  \(X = \bigcup_nA_n\).
\item
  Show that the union of two regions is a region if and only if they have a
  common point.
  \par\smallskip
  For the first implication, \(\Rightarrow\), suppose on the contrary that the
  union of two regions is a region and they have no point in common.
  Let \(A\) and \(B\) be these two nonempty regions.
  Since they share no point in common,
  \(A\cap\bar{B} = B\cap\bar{A} = \varnothing\).
  Therefore, \(A\) and \(B\) are separated so they cannot be a region.
  We have reached a contradiction so if the union of two regions is a region,
  then they have a point in common.
  For the finally implication, suppose they have a point in common and the
  union of two regions is not a region.
  Since the union of two regions is not a region, the regions are separated.
  Let \(A\) and \(B\) be two nonempty separated regions.
  Since \(A\) and \(B\) are separated,
  \(A\cap\bar{B} = B\cap\bar{A} = \varnothing\); therefore, \(A\) and \(B\)
  cannot have a point in common.
  We have reached contradiction so if they have a point in common, then the
  union of two regions is are a region.
\item
  \label{3.1.3.3}
  Prove that the closure of a connected set is connected.
  \par\smallskip
  Let \(T\) be a topological space such that \(E,\bar{E}\subset T\).
  Let \(E\) be a connected set and suppose \(\bar{E}\) is separated; that is,
  \(\bar{E} = A\cup B\) where \(A,B\) are relatively open in \(\bar{E}\),
  nonempty, and disjoint sets.
  Then there exists open sets \(U,V\) in \(T\) such that \(A = U\cap\bar{E}\)
  and \(B = V\cap\bar{E}\).
  Now, \(A\subset U\), \(B\subset V\), and \(U,V\neq\varnothing\).
  Therefore, \(U\cap E\neq\varnothing\neq V\cap E\) so \(U\cap E\) and
  \(V\cap E\) are nonempty, disjoint sets.
  Then \(E = U\cup V\) so \(E\) is separated.
  We have, thus, reached a contradiction and the closure of connected set is
  also connected.
\item
  Let \(A\) be the set of points \((x,y)\in\mathbb{R}^2\) with \(x = 0\),
  \(\lvert y\rvert\leq 1\), and let \(B\) be the set with \(x > 0\),
  \(y = \sin(1/x)\).
  Is \(A\cup B\) connected?
  \par\smallskip
  This is known as the topologist's sine curve.
  \begin{figure}[H]
    \centering
    \includestandalone[width = 3in, mode = image]{Tikz/ch3sec13prob4}
    \caption{Topologist's sine curve plotted on the domain \(x\in(0,1]\).}
    \label{ch3sec13prob4}
  \end{figure}
  Let \(S = A\cup B\).
  We claim that the closure of \(B\) in \(\mathbb{R}^2\) is \(\bar{B} = S\).
  Let \(x\in S\).
  If \(x\in B\), then take a constant sequence \(\{x,x,\ldots\}\).
  If \(x\in A\), then \(x = (0,y)\) for \(\lvert y\rvert\leq 1\) or said
  another way \(y = \sin(\theta)\) for \(\theta\in[-\pi,\pi]\).
  We can write \(y = \sin(\theta)\) as \(y = \sin(\theta + 2k\pi)\) for
  \(k\in\mathbb{Z}^+\).
  Let \(x_k = 1/(\theta + 2k\pi) > 0\).
  Then \(y = \sin(1/x_k)\).
  Now \(\{x_k\}\to 0\) when \(k\to\infty\).
  Then \((x_k,\sin(1/x_k)) = (x_n,y)\to (0,y)\in\bar{B}\) since
  \(\lvert y\rvert\leq 1\).
  Therefore, \(S\subset\bar{B}\).
  Let \(\{(x_n,y_n)\}\in S\) such that \(\{(x_n,y_n)\}\to
  (x,y)\in\mathbb{R}^2\).
  Then \(\lim_{n\to\infty}x_n = x\) and \(\lim_{n\to\infty}y_n = y\).
  From the definition of the sets, \(x\geq 0\) and \(\lvert y\rvert\leq 1\) so
  \(\lvert y\rvert = \lim_{n\to\infty}\lvert y_n\rvert\leq 1\).
  If \(x = 0\), then \((0,y)\in S\) since \(\lvert y\rvert\leq 1\).
  Suppose \(x > 0\).
  Then there exist \(m > N\) such that \(x_m > 0\) for all \(m > N\) so
  \((x_n,y_n)\in B\).
  Let \(y_n = \sin(1/x_n)\) since \((x_n,y_n)\in B\).
  Notice that for \(z\in(0,\infty)\), \(\sin(1/z\) is continuous.
  Since \(\{x_n\}\to x\) and \(y_n = \sin(1/x_n)\), we have
  \[
  y = \lim_{n\to\infty}y_n = \lim_{n\to\infty}\sin(1/x_n) = \sin(1/x).
  \]
  Thus, \((x,y)\in\bar{B}\subset S\) so \(\bar{B} = S\).
  Since \(A\cap\bar{B} = A\cap S = A\neq\varnothing\), \(S\) is connected.
  However, \(S\) is not path connected.
  That is, being connected doesn't imply path connectedness.
  \par\smallskip
  Suppose \(S\) is path connected and there exists an \(f\colon [0,1]\to S\)
  such that \(f(0)\in B\) and \(f(1)\in A\).
  Since \(A\) is path connected, suppose \(f(1) = (0,1)\).
  Let \(\epsilon = 1/2 > 0\).
  By continuity, for \(\delta > 0\), \(\lvert f(t) - (0,1)\rvert < 1/2\)
  whenever \(1 - \delta\leq t\leq 1\).
  Since \(f\) is continuous, the image of \(f([1 - \delta,1])\) is connected.
  Let \(f(1- \delta) = (x,y)\).
  Consider the composite of \(f\colon [1 - \delta,1]\to\mathbb{R}^2\) and its
  projection on the \(x\)-axis.
  Since both maps are continuous as well as their composite, the image of the
  composite map is a connected subset of \(\mathbb{R}^1\) which contains zero
  and \(x\).
  Now zero is the \(x\)-coordinate of \(f(1)\) and \(x\) the \(x\)-coordinate
  of \(f(1 - \delta)\).
  Since \(\mathbb{R}^1\) is convex, connected sets are intervals.
  Then the set of \(x\)-coordinates for \(f(1 - \delta)\) is \(x_0\in[0,x]\).
  For \(x_0\in(0,x]\), there exists \(t\in[1 - \delta,1]\) such that
  \(f(t) = (x_0,\sin(1/x_0))\).
  If \(x_0 = 1/(2k\pi - \pi/2)\) for \(k\gg 1\), then \(0 < x_0 < x\).
  Now \(1/x_0 = \pi(4k - 1)/2\) which is a \(2\pi\) mutliple of \(-\pi/2\) for
  all \(k\).
  Therefore, \(\sin(1/x_0) = -1\) so
  \((x_0, \sin(1/x_0)) = (1/(2k\pi - \pi/2),-1)\) for some
  \(t\in[1 - \delta,1]\) which lies within a distance of \(\epsilon = 1/2\) of
  \((0,1)\).
  However, the distance between \((1/(2k\pi - \pi/2),-1)\) and \((0,1)\) for
  large \(k\) is greater than \(1\) which is a contradiction.
  Thus, \(S\) cannot be path connected.
\item
  Let \(E\) be the set of points \((x,y)\in\mathbb{R}^2\) such that
  \(0\leq x\leq 1\) and either \(y = 0\) or \(y = 1/n\) for some positive
  integer \(n\).
  What are the components of \(E\)?
  Are they all closed?
  Are they relatively open?
  Verify that \(E\) is not locally connected.
\item
  Prove that the components of a closed set are closed (use \cref{3.1.3.3}).
\item
  A set is said to be \textit{discrete} if all its points are isolated.
  Show that a discrete set in a separable metric space is countable.
\end{exercise}

\subsection{Compactness}

\begin{exercise}
\item
  Given an alternate proof of the fact that every bounded sequence of complex
  numbers has a convergent subsequence (for instance by use of the limes
  inferior).
\item
  Show that the Heine-Borel property can also be expressed in the following
  manner: Every collection of closed sets with an empty intersection contains
  a finite subcollection with an empty intersection.
  \par\smallskip
  The statement above is equivalent to: A collection \(\mathcal{F}\) of closed
  subsets of a topological space \((X,\mathcal{T})\) has the finite
  intersection property if \(\cap\mathcal{F}_{\alpha}\neq\varnothing\) for all
  finite subcollections \(\mathcal{F}_{\alpha}\subset\mathcal{F}\).
  Show that \((X,\mathcal{T})\), a topological space, is compacy if and only
  if every family of closed sets \(\mathcal{F}\subset\mathcal{P}(X)\) having
  the finite intersection property satisfies
  \(\cap\mathcal{F}\neq\varnothing\).
  \par\smallskip
  Let \(\mathcal{F} = \{F_{\alpha}\colon\alpha\in A\}\) be a collection of
  closed sets in \(X\).
  Now
  \[
  \bigcap_{\alpha}F_{\alpha} = \varnothing\iff\bigcup_{\alpha}F_{\alpha}^c = X.
  \]
  Therefore, the set \(\bigcup_{\alpha}F_{\alpha}^c\) is an open cover of
  \(X\) since \(F_{\alpha}\) is closed.
  If the intersection of the set \(\{F_{\alpha_n}\}\) is empty for a finite
  \(n\), then \(\bigcup_{\alpha}F_{\alpha_n}^c\) is a finite subcover of \(X\).
  Then every open cover of \(X\) has a finite subcover if and only if every
  collection of closed sets with an empty intersection has a finite
  subcollection with an empty intersection.
  Thus, \(X\) is compact if and only if every collection of closed sets with
  the finite intersection property has a nonempty intersection.
\item
  Use compactness to prove that a closed bounded set of real numbers has a
  maximum.
  \par\smallskip
  Since we are dealing with a set of real numbers, we are speaking of compact
  metric spaces.
  A subset \(E\) of a metric space \(X\) is compact if and only if every
  sequence in \(E\) has convergent subsequence in \(E\) (sequentially
  compact).
  \par\smallskip
  First, we will show \(\Rightarrow\) by contradiction.
  Let \(\{x_n\}\) be a sequence in \(E\).
  Suppose \(\{x_n\}\) doesn't have a convergent subsequence in \(E\).
  Then for \(x\in E\), there exists \(\epsilon > 0\) such that
  \(x_n\in N_{\epsilon}(x)\) for only finitely many \(n\).
  Then \(N_{\epsilon}(x)\) would be an open cover of \(E\) which has no finite
  subcover.
  Therefore, \(E\) couldn't be compact contradicting the premise.
  Thus, if \(E\) is compact metric space, then \(E\) is sequentialy compact.
  In order to prove \(\Leftarrow\), we need to prove that a sequentially
  compact set contains a countable dense subset.
  \par\smallskip
  \textbf{Lemma} \(\mathbold{3.1.4.1:}\) A sequentially compact set contains
  a countable dense subset (separable space).
  \par\smallskip
  Let \(A\) be an infinite sequentially compact set.
  Since \(A\) is sequential compact, \(A\) is bounded; otherwise, we would have
  nonconvergent subsequences in \(A\).
  Let \(\{y_n\}\) be a dense sequence in \(A\).
  Choose \(y_1,y_2,\ldots,y_n\) of \(\{y_n\}\).
  Let \(\delta_n = \sup_{y\in A}\min_{k\leq n}d(y,y_n) > 0\).
  Let \(y_{n + 1}\) be such that \(d(y_{n + 1},y_k)\geq\delta/2\) for
  \(k = 1,\ldots,n\).
  Since \(\{y_n\}\) has a convergent subsequence, for all \(\epsilon > 0\)
  there exist \(m,n\in\mathbb{Z}^+\) such that \(d(y_m,y_n) < \epsilon\).
  Then
  \[
  d(y,y_{n - 1}) < \delta_{n - 1}/2 < d(y_m,y_n) < \epsilon\iff
  \delta_{n - 1} < 2\epsilon
  \]
  Thus, all \(y\in A\) is in \(2\epsilon\) of \(y_k\) for \(k < n\).
  Since \(\epsilon > 0\) and arbitrary, \(\{y_n\}\) is dense in \(A\) because
  every nonempty open set contains at least one element of the sequence.
  \par\smallskip
  Now for \(\Leftarrow\).
  Let \(F_{\alpha}\) be an open cover \(E\) and let \(\{y_n\}\) be a dense
  sequence.
  Let \(r\in\mathbb{Q}\) and let \(G\) be the family of neighborhoods,
  \(N_r(y_n)\), that are contained in \(F_{\alpha}\).
  Since \(\mathbb{Q}\) is countable, \(G\) is countable.
  Let \(x\in E\) and \(x\in F_{\alpha}\).
  Then \(N_{\epsilon}(x)\subset F_{\alpha}\) for \(\epsilon > 0\).
  Since \(\{y_n\}\) is dense in \(E\), by lemma \(3.1.4.1\),
  \(d(y,y_n) < \epsilon/2\) for some \(n\).
  For all \(r\in\mathbb{Q}\), \(d(y_n,y) < r < \epsilon - d(y_n,y)\).
  Then \(x\in N_r(y_n)\subset N_{\epsilon}(x)\subset G\).
  Since \(x\in N_r(y_n)\in G\) and \(G\) is countable, we can find a finite
  subcover of \(G\).
  Replace each \(G\) by \(F_{\alpha}\) where \(G\subset F_{\alpha}\) for some
  \(\alpha\).
  Then this set of \(F_{\alpha}\) is a finite subcover.
  Thus, if \(E\) is a sequentially compact metric space, then \(E\) is compact.
\item
  If \(E_1\supset E_2\supset\cdots\) is a decreasing sequence of nonempty
  compact sets, then the intersection \(\bigcap_1^{\infty}E_n\) is not empty
  (Cantor's lemma).
  Show by example that this need not be true if the sets are merely closed.
  \par\smallskip
  Consider the topological space \(\mathbb{R}^1\).
  Let \(E_n = \{n\in\mathbb{Z}^{\geq 0}\colon [n,\infty)\}\).
  Then \(E_n\supset E_{n + 1}\supset\cdots\).
  Since \((-\infty,n)\) is open in \(\mathbb{R}^1\),
  \([n,\infty) = (-\infty,n)^c\) is closed.
  The infinite intersection of \(E_n\) is
  \[
  \bigcap_{n = 1}^{\infty}E_n = \bigcap_{n = 1}^{\infty}[n, \infty) =
  \varnothing.
  \]
  Thus, the statement isn't true if we consider only closed sets.
\item
  Let \(S\) be the set of all sequences \(x = \{x_n\}\) of real numbers such
  that only a finite number of the \(x_n\) are \(\neq 0\).
  Define \(d(x,y) = \max\lvert x_n - y_n\rvert\).
  Is the space complete?
  Show that the \(\delta\)-neighborhoods are not totally bounded.
\end{exercise}

\subsection{Continuous Functions}

\subsection{Topological Spaces}

\section{Conformality}

%%% Local Variables:
%%% mode: latex
%%% TeX-master: t
%%% End:
