\chapter{Analytic Functions as Mappings}

\section{Elementary Point Set Topology}

\subsection{Sets and Elements}

\subsection{Metric Spaces}

\begin{enumerate}[ref = \theenumi{}]
\item
  \label{3.1.2.1}
  If \(S\) is a metric space with distance function \(d(x,y)\), show that \(S\)
  with the distance function \(\delta(x,y) = d(x,y)/[1 + d(x,y)]\) is also a
  metric space.
  The latter space is bounded in the sense that all distances lie under a fixed
  bound.
  \par\smallskip
  Since \(d(x,y)\) is a metric on \(S\), \(d(x,y)\) satifies
  \begin{subequations}
    \begin{align}
      d(x, y) & \geq 0,\text{  and zero only when \(x = y\)}\label{3.1.2.1a}\\
      d(x, y) & = d(y, x)\label{3.1.2.1b}\\
      d(x, z) & \leq d(x, y) + d(y, z)\label{3.1.2.1c}
    \end{align}
  \end{subequations}
  By \cref{3.1.2.1a}, for \(x = y\), \(d(x,y) = 0\) so
  \(\delta(x,y) = 0/1 = 0\), and when \(x\neq y\), \(d(x,y) > 0\) so
  \(\delta(x,y) > 0\) since a positive number divided by a positive number is
  positive.
  We have that \(\delta(x,y)\geq 0\) and equal zero if and only if \(x = y\).
  By \cref{3.1.2.1b}, we have
  \begin{align*}
    \delta(x, y) & = \frac{d(x, y)}{1 + d(x, y)}\\
                 & = \frac{d(y, x)}{1 + d(y, x)}\\
                 & = \delta(y, x)
  \end{align*}
  For the triangle inequality, we have
  \begin{align*}
    \frac{d(x, z)}{1 + d(x, z)}
    & \leq \frac{d(x, y)}{1 + d(x, y)} + \frac{d(y, z)}{1 + d(y, z)}\\
    \intertext{Let's multiple through by the product of all three
    denominators.
    After simplifying, we obtain}
    d(x, z) & \leq d(x, y) + d(y,z) + 2d(x, y)d(y, z) + d(x, y)d(y, z)d(x, z)
  \end{align*}
  We have already shown that \(d(x,y)\geq 0\) and zero if and only if
  \(x = y\).
  If \(x = y = z\), the triangle inequality is vacuously true.
  When \(x\neq y\neq z\), the triangle inequality follows since each distance
  is positive and \cref{3.1.2.1c}; that is,
  \[
  \delta(x, z)\leq \delta(x, y) + \delta(y, z).
  \]
\item
  Suppose that there are given two distance functions \(d(x,y)\) and
  \(d_1(x,y)\) on the same space \(S\).
  They are said to be equivalent if they determine the same open sets.
  Show that \(d\) and \(d_1\) are equivalent if to every \(\epsilon > 0\) there
  exists a \(\delta > 0\) such that \(d(x,y) < \delta\) implies
  \(d_1(x,y) < \epsilon\), and vice versa.
  Verify that this condition is fulfilled in \cref{3.1.2.1}.
\item
  Show by strict application of the definition that the closure of
  \(\lvert z - z_0\rvert < \delta\) is \(\lvert z - z_0\rvert\leq\delta\).
\item
  If \(X\) is the set of complex numbers whose real and imaginary parts are
  rational, what is \(\operatorname{Int} X, \bar{X}, \partial X\)?
\item
  It is sometimes typographically simpler to write \(X'\) for \(\sim X\).
  With this notation, how is \(X^{'-'}\) related to \(X\)?
  Show that \(X^{-'-'-'-'} = X^{-'-'}\).
\item
  A set is said to be discrete if all its points are isolated.
  Show that a discrete set in \(\mathbb{R}\) or \(\mathbb{C}\) is countable.
\end{enumerate}
%%% Local Variables:
%%% mode: latex
%%% TeX-master: t
%%% End:
