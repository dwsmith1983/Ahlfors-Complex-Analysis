\chapter{Analytic Functions as Mappings}

\section{Elementary Point Set Topology}

\subsection{Sets and Elements}

\subsection{Metric Spaces}

\begin{exercise}[ref = \arabic{exercisei}]
\item
  \label{3.1.2.1}
  If \(S\) is a metric space with distance function \(d(x,y)\), show that \(S\)
  with the distance function \(\delta(x,y) = d(x,y)/[1 + d(x,y)]\) is also a
  metric space.
  The latter space is bounded in the sense that all distances lie under a fixed
  bound.
  \par\smallskip
  Since \(d(x,y)\) is a metric on \(S\), \(d(x,y)\) satifies
  \begin{subequations}
    \begin{align}
      d(x, y) & \geq 0,\text{  and zero only when \(x = y\)}\label{3.1.2.1a}\\
      d(x, y) & = d(y, x)\label{3.1.2.1b}\\
      d(x, z) & \leq d(x, y) + d(y, z)\label{3.1.2.1c}
    \end{align}
  \end{subequations}
  By \cref{3.1.2.1a}, for \(x = y\), \(d(x,y) = 0\) so
  \(\delta(x,y) = 0/1 = 0\), and when \(x\neq y\), \(d(x,y) > 0\) so
  \(\delta(x,y) > 0\) since a positive number divided by a positive number is
  positive.
  We have that \(\delta(x,y)\geq 0\) and equal zero if and only if \(x = y\).
  By \cref{3.1.2.1b}, we have
  \begin{align*}
    \delta(x, y) & = \frac{d(x, y)}{1 + d(x, y)}\\
                 & = \frac{d(y, x)}{1 + d(y, x)}\\
                 & = \delta(y, x)
  \end{align*}
  For the triangle inequality, we have
  \begin{align*}
    \frac{d(x, z)}{1 + d(x, z)}
    & \leq \frac{d(x, y)}{1 + d(x, y)} + \frac{d(y, z)}{1 + d(y, z)}\\
    \intertext{Let's multiple through by the product of all three
    denominators.
    After simplifying, we obtain}
    d(x, z) & \leq d(x, y) + d(y,z) + 2d(x, y)d(y, z) + d(x, y)d(y, z)d(x, z)
  \end{align*}
  We have already shown that \(d(x,y)\geq 0\) and zero if and only if
  \(x = y\).
  If \(x = y = z\), the triangle inequality is vacuously true.
  When \(x\neq y\neq z\), the triangle inequality follows since each distance
  is positive and \cref{3.1.2.1c}; that is,
  \[
  \delta(x, z)\leq \delta(x, y) + \delta(y, z).
  \]
\item
  Suppose that there are given two distance functions \(d(x,y)\) and
  \(d_1(x,y)\) on the same space \(S\).
  They are said to be equivalent if they determine the same open sets.
  Show that \(d\) and \(d_1\) are equivalent if to every \(\epsilon > 0\) there
  exists a \(\delta > 0\) such that \(d(x,y) < \delta\) implies
  \(d_1(x,y) < \epsilon\), and vice versa.
  Verify that this condition is fulfilled in \cref{3.1.2.1}.
  \par\smallskip
  Let \(\epsilon,\delta > 0\) be given.
  We can write
  \(\delta(x,y) = \frac{d(x,y)}{1 + d(x,y)} = 1 - \frac{1}{1 + d(x,y)}\).
  We need to find a \(\delta\) such that whenver \(d(x,y) < \delta\),
  \(\delta(x,y) < \epsilon\).
  \begin{align*}
    1 - \frac{1}{1 + d(x, y)} & < \epsilon\\
    d(x, y) & < \frac{\epsilon}{1 - \epsilon}
  \end{align*}
  Let \(\delta = \frac{\epsilon}{1 - \epsilon}\).
  For \(\epsilon < 1\), if \(d(x,y) < \delta = \frac{\epsilon}{1 - \epsilon}\),
  then
  \[
  \delta(x,y) = 1 - \frac{1}{1 + d(x, y)} < 1 -
  \frac{1}{1 + \frac{\epsilon}{1 - \epsilon}} = \epsilon.
  \]
  If \(\epsilon > 1\),
  \(\delta(x,y) = \frac{d(x,y)}{1 + d(x,y)} < 1 < \epsilon\).
  For the reverse implication, we need to find a \(\delta\) such that
  \(d(x,y) < \epsilon\).
  Let \(\delta = \frac{\epsilon}{1 + \epsilon}\).
  For any \(\epsilon > 0\), if
  \(\delta(x,y) < \delta = \frac{\epsilon}{1 + \epsilon}\), then
  \begin{align*}
    \frac{d(x, y)}{1 + d(x, y)} & < \frac{\epsilon}{1 + \epsilon}\\
    d(x, y)(1 + \epsilon) & < \epsilon + \epsilon d(x, y)\\
    d(x, y) & < \epsilon
  \end{align*}
  as was needed to be shown.
  Therefore, \(d(x,y)\) and \(\delta(x,y)\) are equivalent metrics on \(S\).
\item
  Show by strict application of the definition that the closure of
  \(\lvert z - z_0\rvert < \delta\) is \(\lvert z - z_0\rvert\leq\delta\).
\item
  If \(X\) is the set of complex numbers whose real and imaginary parts are
  rational, what is \(\operatorname{Int} X, \bar{X}, \partial X\)?
\item
  It is sometimes typographically simpler to write \(X'\) for \(\sim X\).
  With this notation, how is \(X^{'-'}\) related to \(X\)?
  Show that \(X^{-'-'-'-'} = X^{-'-'}\).
\item
  A set is said to be discrete if all its points are isolated.
  Show that a discrete set in \(\mathbb{R}\) or \(\mathbb{C}\) is countable.
  \par\smallskip
  Let \(S\) be a dsicrete set in \(\mathbb{R}\) or \(\mathbb{C}\).
  If \(z\in S\), then for some \(\epsilon_i > 0\), for \(i\in\mathbb{Z}\),
  \(N_{\epsilon_i}(z_i)\) is the \(i\)-th neighborhood of \(z_i\).
  Since \(S\) is discrete, there exists an \(\epsilon\) for each \(i\) such
  that the only point in \(N_{\epsilon_i}(z_i)\) is \(z_i\).
  Let \(\epsilon_i\) be this \(\epsilon\). 
  Consider the following function
  \[
  f(i) =
  \begin{cases}
    0, & i = 0\\
    1, & i = 1\\
    2i, & i > 0\\
    2(-i) + 1, & i < 0
  \end{cases}
  \]
  We have put \(i\) in a one-to-one correspondence with \(\mathbb{Z}^+\).
  Therefore, \(S\) is countable.
\item
  Show that the accumulation points of any set form a closed set.
  \par\smallskip
  Let \(E\) be a set.
  Then \(E'\) is the set of accumulation (limit) points.
  If \(z_i\) is a limit point, \(z_i\in E'\).
  Now, \(z_i\) are limit points of \(E\) as well.
  Then \(\{z_i\}\to z\) where \(z\in\bar{E}\).
  Therefore, \(z\in E'\) so \(E'\) is closed.
\end{exercise}

\subsection{Connectedness}

\begin{exercise}[ref = \arabic{exercisei}]
\item
  If \(X\subset S\), show that the relatively open (closed) subsets of \(X\)
  are precisely those sets that can be expressed as the intersection of \(X\)
  with an open (closed) subsets of \(S\).
  \par\smallskip
  Let \(\{U_{\alpha}\}\) be the open sets of \(S\) such that
  \(\bigcup_{\alpha}U_{\alpha} = S\).
  Then
  \(X = X\cap\bigcup_{\alpha}U_{\alpha} = \bigcup_{\alpha}(X\cap U_{\alpha})\).
  Let \(\{A_n\} = \{X\cap U_{\alpha}\}\).
  Then \(A_n\) is relatively open in \(X\) since \(A_n\) belongs to the
  topology of \(X\); that is, for each \(n\), \(A_n\subset X\) and
  \(X = \bigcup_nA_n\).
\item
  Show that the union of two regions is a region if and only if they have a
  common point.
  \par\smallskip
  For the first implication, \(\Rightarrow\), suppose on the contrary that the
  union of two regions is a region and they have no point in common.
  Let \(A\) and \(B\) be these two nonempty regions.
  Since they share no point in common,
  \(A\cap\bar{B} = B\cap\bar{A} = \varnothing\).
  Therefore, \(A\) and \(B\) are separated so they cannot be a region.
  We have reached a contradiction so if the union of two regions is a region,
  then they have a point in common.
  For the finally implication, suppose they have a point in common and the
  union of two regions is not a region.
  Since the union of two regions is not a region, the regions are separated.
  Let \(A\) and \(B\) be two nonempty separated regions.
  Since \(A\) and \(B\) are separated,
  \(A\cap\bar{B} = B\cap\bar{A} = \varnothing\); therefore, \(A\) and \(B\)
  cannot have a point in common.
  We have reached contradiction so if they have a point in common, then the
  union of two regions is are a region.
\item
  \label{3.1.3.3}
  Prove that the closure of a connected set is connected.
\item
  Let \(A\) be the set of points \((x,y)\in\mathbb{R}^2\) with \(x = 0\),
  \(\lvert y\rvert\leq 1\), and let \(B\) be the set with \(x > 0\),
  \(y = \sin(1/x)\).
  Is \(A\cup B\) connected?
\item
  Let \(E\) be the set of points \((x,y)\in\mathbb{R}^2\) such that
  \(0\leq x\leq 1\) and either \(y = 0\) or \(y = 1/n\) for some positive
  integer \(n\).
  What are the components of \(E\)?
  Are they all closed?
  Are they relatively open?
  Verify that \(E\) is not locally connected.
\item
  Prove that the components of a closed set are closed (use \cref{3.1.3.3}).
\item
  A set is said to be \textit{discrete} if all its points are isolated.
  Show that a discrete set in a separable metric space is countable.
\end{exercise}
%%% Local Variables:
%%% mode: latex
%%% TeX-master: t
%%% End:
